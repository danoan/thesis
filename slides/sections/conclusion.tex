\section{Conclusion}

\begin{frame}
{Conclusion}
{Summary of models}

\onslide<1->{
\begin{table}[H]
\footnotesize
\centering
\begin{tabular}{r|ccccc}
\multirow{2}{*}{Model} & \multirow{2}{*}{Implementation} & Running & Free & Constrained & Image\\
& & time & elastica & elastica & term\\
\hline
LocalSearch & medium & \negative{slow} & \positive{yes(\textbf{opt})} & \positive{yes} & \negative{no}\\
FlipFlow  & \negative{hard} & acceptable & \positive{yes} & \negative{no} & \positive{yes}\\
( BalanceFlow ) & medium & acceptable & \positive{yes} & \negative{no} & \positive{yes}\\
GraphFlow  & \positive{easy} & \positive{fast} & \positive{yes(\textbf{opt})} & \positive{yes} & \positive{yes}
\end{tabular}
\caption{\textbf{Models summary.} The qualitative attributes are relative, e.g., the GraphFlow presents the lowest running time while LocalSearch presents the highest.}
\end{table}}

\onslide<2->{
\begin{figure}
\center
\captionsetup{type=table}
\footnotesize
\begin{tabular}{|l|c|c|c|c|c|}
\hline
& Pixels & LocalSearch & FlipFlow & BalanceFlow & GraphFlow \\
\hline
Triangle & 8315 & \negative{4.8s/it} & 0.4s/it & 0.38s/it & \positive{0.14s/it}\\
Square & 12769 & \negative{2s/it} & 0.51s/it & 0.47s/it & \positive{0.12s/it}\\
Ellipse  & 10038 & \negative{3.1s/it} & 0.64s/it & 0.57s/it & \positive{0.1s/it} \\
Flower & 26321 & \negative{12.3s/it} & 1.23s/it & 0.94s/it & \positive{0.14s/it}\\
Bean  & 25130 & \negative{6.4s/it} & 1.2s/it & 1.17s/it & \positive{0.16s/it}\\
\hline
\end{tabular}
\caption{\textbf{Free elastica running times.} Running time and input size for the free elastica experiment.}
\end{figure}}

\end{frame}

\begin{frame}
{Conclusion}
{Summary of models}

\begin{itemize}
\item{We achieved global optimum elastica with a digital model.}
\pause
\item{GraphFlow is extendable (suitable for data terms) and our fastest model.}
\pause
\item{Contour completion is achieved in some cases.}
\pause
\end{itemize}

\textbf{Pros}
\begin{itemize}
\item{Topology is flexible.}
\pause
\item{Easily parallelizable.}
\pause
\item{Flexibility of neighborhood of shapes.}
\pause
\end{itemize}

\textbf{Cons}
\begin{itemize}
\item{Susceptible to bad local minimum (we can escape it with a proper definition of the neighborhood).}
\end{itemize}

\end{frame}

\begin{frame}
{Conclusion}
{Perspectives}

\begin{itemize}
\item{\textbf{GraphFlow and perimeter}: enrich the cost function of GraphFlow with the weights defined in~\mycite{boykov03geodesics}. }
\pause
\item{\textbf{Different neighborhoods}: random, linear extension.}
\pause
\item{\textbf{Dynamic radius}: use the parameter free Maximal Digital Circular Arcs estimator of curvature to adapt the estimation disk radius to use.}
\pause
\item{\textbf{Multiresolution}: Improve running time; or improve estimator precision.}
\pause
\item{\textbf{Image analysis applications}: Make an objective comparison of our method and competitive ones (e.g. study quantitative measurements such as the ratio of inflexion points for the contour correction application) .}
\pause
\item{\textbf{Global formulation and multigrid convergent estimators}: Do there exist a practicable model for elastica?}
\end{itemize}

\end{frame}

\begin{frame}
\huge
\center
Thank you!
\end{frame}
