\section{Non-submodular elastica}

\begin{frame}
\begin{center}
\huge
A quadratic non-submodular formulation for elastica
\end{center}
\end{frame}

\begin{frame}
	{Non-submodular elastica}	
	{Difficulties with a global model}
\begin{minipage}[t][0.6\textheight][t]{1\textwidth}
\begin{minipage}{0.35\textwidth}
\only<1>{
\includegraphics[scale=0.5]{figures/non-submodular-elastica/global/issues-2.png}
}
\only<2>{
\includegraphics[scale=0.5]{figures/non-submodular-elastica/global/issues-3.png}
}
\only<3>{
\includegraphics[scale=0.5]{figures/non-submodular-elastica/global/issues-4.png}
}
\only<4->{
\includegraphics[scale=1]{figures/non-submodular-elastica/global/topological-constraints.png}
}
\end{minipage}
%
%
\begin{minipage}{0.64\textwidth}
\onslide<2->{
\begin{itemize}
	\item{Center of the estimation disk}
	\onslide<3->{\item{Pixel counting and estimation of curvature squared}}
	\onslide<4->{\item{Topological constraints}}	
	\onslide<5->{\item{Third order non-convex binary}}		
	\onslide<6->{\item{Level 1 linearization: non semi-definite positive matrix}}
	\onslide<7->{\item{Level 2 linearization: $O(m^3)$ variables}}
\end{itemize}}
\end{minipage}	
\end{minipage}
%
%
\begin{minipage}[t][0.25\textheight][t]{1\textwidth}
\only<2>{
\begin{align*}
	\sum_{\ell_i \in \mathcal{L}}{ \vec{y}_i \left(\; \alpha + \beta \hat{\kappa}_{r}^2(D,\ell_i) \; \right)}\\\nonumber
\end{align*}}
\only<3>{
\begin{align*}
&\sum_{\ell_i \in \mathcal{L}}{ \vec{y}_i \left(\; \alpha + \frac{9}{r^6}\beta \big(c^2 - 2c\vec{A}_i^T\vec{x} + \vec{x}^T\vec{A}_i\vec{A}_i^T\vec{x}\big)\right)}\\
&\text{subject to} \quad \vec{x} \in \{0,1\}^m, \vec{y} \in \{0,1\}^n.
\end{align*}}
\only<4->{
\begin{align*}
&\sum_{\ell_i \in \mathcal{L}}{ \vec{y}_i \left(\; \alpha + \frac{9}{r^6}\beta \big(c^2 - 2c\vec{A}_i^T\vec{x} + \vec{x}^T\vec{A}_i\vec{A}_i^T\vec{x}\big)\right)}\\
&\text{subject to} \quad \vec{x} \in \{0,1\}^m, \vec{y} \in \{0,1\}^n, T(\vec{x},\vec{y}).
\end{align*}}
\end{minipage}

\end{frame}

\begin{frame}
{Non-submodular elastica}
{Simplification}
\center
$\hat{\kappa}(p) = \frac{3}{r^3}\left( \frac{\pi r^2}{2} - | B_r(p) \cap X | \right )$
\begin{minipage}[t][0.5\textheight]{1\textwidth}
\center
\only<1>{
\includegraphics[scale=0.5]{figures/non-submodular-elastica/current-contour.png}}
\only<2>{
\includegraphics[scale=0.1]{figures/non-submodular-elastica/before-opt.png}\hspace{2em}
\includegraphics[scale=0.1]{figures/non-submodular-elastica/shape-opt-ball-after.png}}
\end{minipage}

\begin{itemize}
\item{Define an optimization band (yellow) as the inner contour of the shape, denoted $I$.}
\item{Set pixels such that the curvature estimation is reduced}
\end{itemize}
\end{frame}

\begin{frame}
{Non-submodular elastica}
{Simplification}
\center
$\hat{\kappa}(p) = \frac{3}{r^3}\left( \frac{\pi r^2}{2} - | B_r(p) \cap X | \right )$
\begin{minipage}[t][0.5\textheight]{1\textwidth}
\center
\includegraphics[scale=0.1]{figures/non-submodular-elastica/before-opt.png}\hspace{2em}
\only<1-2>{
\includegraphics[scale=0.1]{figures/non-submodular-elastica/shape-opt-ball-after.png}}
\only<3>{
\includegraphics[scale=0.1]{figures/non-submodular-elastica/shape-opt-after-inverted.png}}
\end{minipage}

\begin{itemize}
\item{Optimization identifies zones of shortage (convex) or abundance (concave) of pixels.}
\onslide<2->{
\item{$x=1 \rightarrow$ Zone of shortage of pixels (convex) $\rightarrow$ Estimator disk should be shifted towards the interior $\rightarrow$ This pixel does not belong to the next contour. }}
\end{itemize}
\end{frame}

\begin{frame}
{Non-submodular elastica}
{FlipFlow}
\begin{minipage}[t][0.6\textheight][t]{1\textwidth}
\footnotesize
\only<1-5>{
\begin{align*}
  E_{(\vec{\theta},m)}^{flip}( \highlight{2}{1,3-5}{ \Ds^{(k)},X^{(k)} } ) =& \sum_{ x_j \in X^{(k)}}{ \alpha s(x_j)} +  \sum_{ p \in \highlight{3}{1-2,4-}{R_m(\Ds^{(k)})}}{\beta \hat{\kappa}(p)^2}\\ 
  \onslide<4->{=& \sum_{ x_j \in X^{(k)}}{ \alpha \highlight{5}{1-4,6-}{s(x_j)}} \\
  +&\sum_{ \substack{p \in \\ R_m(\Ds^{(k)})}}{ 2c_1 \beta  \Big( { (1/2+ |F_{r}^{(k)}(p)|-c_2) \cdot \sum_{ \substack{ x_j \in \\ X_{r}^{(k)}(p)}}{x_j} } + \sum_{ \substack{j<l, \\ x_j,x_l \in \\ X_{r}^{(k)}(p) } }{x_jx_l} \Big) }}
\end{align*}}
\only<6->{
\begin{align*}
  E_{(\vec{\theta},m)}^{flip}( \highlight{2}{1,3-5}{ \Ds^{(k)},{\highlight{6}{7-}{1-X^{(k)}} }} ) =& \sum_{ x_j \in X^{(k)}}{ \alpha s(x_j)} +  \sum_{ p \in \highlight{3}{1-2,4-}{R_m(\Ds^{(k)})}}{\beta \hat{\kappa}(p)^2}\\ 
  \onslide<4->{=& \sum_{ x_j \in X^{(k)}}{ \alpha \highlight{5}{1-4,6-}{s(x_j)}} \\
  +&\sum_{ \substack{p \in \\ R_m(\Ds^{(k)})}}{ 2c_1 \beta  \Big( { (1/2+ |F_{r}^{(k)}(p)|-c_2) \cdot \sum_{ \substack{ x_j \in \\ X_{r}^{(k)}(p)}}{x_j} } + \sum_{ \substack{j<l, \\ x_j,x_l \in \\ X_{r}^{(k)}(p) } }{x_jx_l} \Big) }}
\end{align*}}
\end{minipage}
%
%
\begin{minipage}[t][0.39\textheight][t]{1\textwidth}
\only<2-3>{
\begin{center}
$\displaystyle
D \subset \Omega \subset \mathbb{Z}^2, \quad
X^{(k)} := \{ \; x_i \in \{0,1\} \; | \; p_i \in I^{(k)} \; \}$
\end{center}}
\only<3>{
\begin{center}
$\displaystyle
R_m(D) := \{ p \; | \; m-1 < d_D(p) \leq m \} \cup \{ p \; | \; -m+1 > d_D(p) \geq -m \}$
\end{center}}
\only<5>{
\begin{center}
\[
  s(x_j)=\sum_{q_i \in \mathcal{N}_4(p_j)}{ t(q_i) }, \quad \text{where } t(q_i) = \left\{\begin{array}{ll}
  (x_j-x_i)^2, & \text{if } q_i \in I^{(k)}\\
  (x_j-1)^2, & \text{if } q_i \in F^{(k)}\\
  (x_j-0)^2, & \text{otherwise. }
  \end{array}\right.
\]
\end{center}}
\only<6>{
\begin{center}
\[
  s(x_j)=\sum_{q_i \in \mathcal{N}_4(p_j)}{ t(q_i) }, \quad \text{where } t(q_i) = \left\{\begin{array}{ll}
  (x_j-x_i)^2, & \text{if } q_i \in I^{(k)}\\
  (x_j-{\color{blue}0})^2, & \text{if } q_i \in F^{(k)}\\
  (x_j-{\color{blue}1})^2, & \text{otherwise. }
  \end{array}\right.
\]
\end{center}}
\begin{minipage}{0.49\textwidth}
\scriptsize
\only<8->{\transparent{0.4}}
\only<7->{
\begin{center}
Shrink mode (convexities)
\[
\begin{array}{rl}
	a^{(k)} &\leftarrow \displaystyle \argmin_{X^{(k)}} E_{\vec{\Theta},m}^{flip}(D^{(k)},1-X^{(k)});\\[1.5em]
	D^{(k+1)} &\leftarrow F^{(k)} + a^{(k)}.
\end{array}
\]
\end{center}}
\end{minipage}
\begin{minipage}{0.49\textwidth}
\scriptsize
\only<8->{
\begin{center}
Expansion mode (concavities)
\[
\begin{array}{rl}
	a^{(k)} &\leftarrow 	\displaystyle \argmin_{\overline{X}^{(k)}} E_{\vec{\Theta},m}^{flip}(\overline{D}^{(k)},1-\overline{X}^{(k)});\\[1.5em]
	D^{(k+1)} &\leftarrow \overline{ \overline{F}^{(k)} + a^{(k)} }.
\end{array}
\]
\end{center}}
\end{minipage}
\end{minipage}

\end{frame}

\begin{frame}
{Non-submodular elastica}
{Evaluation on farther rings}

\begin{minipage}{0.49\textwidth}
\center
$r=3$\\
\includegraphics[scale=0.2]{figures/non-submodular-elastica/radius-effect/triangle-r3.png}\\[1em]
\includegraphics[scale=0.2]{figures/non-submodular-elastica/radius-effect/flower-r3.png}
\end{minipage}
\begin{minipage}{0.49\textwidth}
\center
$r=5$\\
\includegraphics[scale=0.2]{figures/non-submodular-elastica/radius-effect/triangle-r5.png}\\[1em]
\includegraphics[scale=0.2]{figures/non-submodular-elastica/radius-effect/flower-r5.png}
\end{minipage}

\end{frame}

\begin{frame}
{Non-submodular elastica}
{Evaluation on farther rings}
\begin{tabular}{cccc}
\multicolumn{4}{c}{$r=5$}\\
$m=1$ & $m=3$ & $m=4$ & $m=5$ \\
\includegraphics[scale=0.13]{figures/non-submodular-elastica/level-effect/triangle-l1.png}&
\includegraphics[scale=0.13]{figures/non-submodular-elastica/level-effect/triangle-l3.png}&
\includegraphics[scale=0.13]{figures/non-submodular-elastica/level-effect/triangle-l4.png}&
\includegraphics[scale=0.13]{figures/non-submodular-elastica/level-effect/triangle-l5.png}\\[2em]
\includegraphics[scale=0.13]{figures/non-submodular-elastica/level-effect/flower-l1.png}&
\includegraphics[scale=0.13]{figures/non-submodular-elastica/level-effect/flower-l3.png}&
\includegraphics[scale=0.13]{figures/non-submodular-elastica/level-effect/flower-l4.png}&
\includegraphics[scale=0.13]{figures/non-submodular-elastica/level-effect/flower-l5.png}
\end{tabular}
\end{frame}

\begin{frame}
{Non-submodular elastica}
{Contour correction}
\begin{tabular}{cc}
\includegraphics[scale=0.28]{figures/non-submodular-elastica/contour-correction/gc-seg-airplane.png}&
\includegraphics[scale=0.28]{figures/non-submodular-elastica/contour-correction/corrected-seg-airplane.png}\\[1em]
\includegraphics[scale=0.28]{figures/non-submodular-elastica/contour-correction/gc-seg-panther.png}&
\includegraphics[scale=0.28]{figures/non-submodular-elastica/contour-correction/corrected-seg-panther.png}
\end{tabular}
\end{frame}

\begin{frame}
{Non-submodular elastica}
{Unlabeled ratio}
\includegraphics[scale=0.28]{figures/non-submodular-elastica/level-effect/plot-unlabeled-triangle.png}\hspace{1em}
\includegraphics[scale=0.28]{figures/non-submodular-elastica/level-effect/plot-unlabeled-flower.png}
\begin{itemize}
{\only<3->{\transparent{0.4}}
\onslide<2->{\item{Given that we evaluate the curvature estimator on the initial contour, how to change the pixels in this contour to reduce the difference of inner and outer pixels?}}
}
\onslide<3->{\item{Given that the shape does not change, where the estimation disks should be centered in order to reduce the difference of inner and outer pixels?}}
\end{itemize}

\end{frame}

\begin{frame}
{Non-submodular elastica}
{Balance coefficient}
\begin{minipage}{0.5\textwidth}
\center
\includegraphics[scale=0.2]{figures/non-submodular-elastica/balance-coefficient-zero-level-set.png}
\end{minipage}
\begin{minipage}{0.49\textwidth}
\footnotesize
\begin{itemize}
\item{Balance coefficient}
\begin{align*}
u_r(D,p) &= \left( \frac{\pi r^2}{2} - |B_r(p) \cap D| \right)^2
\end{align*}
\item{White contour: contour of the shape}
\item{Pink contour: zero level set of the balance coefficient}
\end{itemize}
\end{minipage}
\end{frame}

\begin{frame}
{Non-submodular elastica}
{Conclusion}

\begin{itemize}
\item{Change pixels in the contour in order to reduce the difference between inner and outer pixels.}
\item{Quadratic non-submodular energy.}
\pause
\item{Farther rings: better results when disks are evaluated farther from the contour. }
\pause
\item{Pos-processing procedure: contour correction. }
\pause
\item{Unstable hypothesis: fixing the contour (dimension 1) is more sensitive than fixing the shape (dimension 2).}
\item{Make the shape evolve to the zero level set of its balance coefficient.}
\end{itemize}
\end{frame}


%\begin{align*}
%  s(x_{w(p)})=\sum_{q \in \mathcal{N}_4(p)}{ t(q) }, \quad \text{where } t(q) = \left\{\begin{array}{ll}
%  (x_{w(p)}-x_{w(q)})^2, & \text{if } q \in I^{(k)}\\
%  (x_{w(p)}-1)^2, & \text{if } q \in F^{(k)}\\
%  (x_{w(p)}-0)^2, & \text{otherwise. }
%  \end{array}\right.
%\end{align*}