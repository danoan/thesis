\documentclass[10pt,xcolor=svgnames]{beamer}
\usefonttheme{professionalfonts}
\usepackage[utf8]{inputenc}
\usepackage[T1]{fontenc}
\usepackage[french,english]{babel}
\usepackage{graphicx}
\usepackage{caption,subcaption}
\usepackage{multirow}
\captionsetup{compatibility=false}

\usepackage[style=authoryear,maxbibnames=99,maxcitenames=5,backref=true,backend=biber,citestyle=authoryear]{biblatex}
\AtBeginBibliography{\footnotesize}
\addbibresource{main.bib}

\usepackage[linesnumbered,commentsnumbered,ruled,vlined]{algorithm2e}
\usepackage[linewidth=2.5pt,linecolor=black,nobreak=true]{mdframed}
\mdfsetup{frametitlealignment=\center}

\usepackage{amstext,amsmath,amssymb,bm,bbm,graphicx,mathtools,accents}
\usepackage{beamerleanprogress}
\usepackage{tikz}
\usepackage{transparent}

\newcommand{\daniel}[1]{{\color{red}{#1}}}

\newcounter{definition}[section]
\newenvironment{definition}[1]{\refstepcounter{definition} \par\bigskip \noindent  \begin{minipage}[b]{\linewidth} 
\textbf{Definition~\thedefinition(#1):}}{\end{minipage} \par\bigskip}

\newcounter{example}[section]
\newenvironment{example}{\refstepcounter{example} \par\medskip
\noindent  \textbf{Example~\theexample :}}{\medskip}


\newcounter{claim}[section]
\newenvironment{claim}[1]{\refstepcounter{claim} \par\medskip \noindent  \textbf{Claim~\theclaim(#1):}}{\medskip}

\newcounter{lemma}[section]
\newenvironment{lemma}{\refstepcounter{lemma} \par\medskip \noindent  \textbf{Lemma~\thelemma:}}{\medskip}


\newcounter{proposition}[section]
\newenvironment{proposition}[1]{\refstepcounter{proposition} \par\medskip \noindent  \textbf{Proposition~\theproposition(#1):}}{\medskip}

\newcounter{theorem}[section]
\newenvironment{theorem}[1]{\refstepcounter{theorem} \par\medskip \noindent  \textbf{Theorem~\thetheorem(#1):}}{\medskip}


\newcounter{assumption}[section]
\newenvironment{assumption}{ \refstepcounter{assumption} \renewcommand{\theequation}{A.\arabic{assumption}} }{\renewcommand{\theequation}{\arabic{section}.\arabic{equation}}}


\crefname{algocf}{alg.}{algs.}
\Crefname{algocf}{Algorithm}{Algorithms}
\Crefname{appsec}{Appendix}{Appendices}



\DeclareMathOperator*{\argmin}{arg\,min}
\DeclareMathOperator*{\argmax}{arg\,max}
\renewcommand{\vec}[1]{\bm{#1}}
\DeclarePairedDelimiter\norm{\lVert}{\rVert}%
\DeclarePairedDelimiter\bignorm{\Big\lVert}{\Big\rVert}%
\DeclarePairedDelimiter\abs{\lvert}{\rvert}%

\newcommand{\edge}[2]{ ( \overrightarrow{ #1,#2} )}

\newcommand{\sketch}[1]{{\color{red}{#1}}}
\newcommand{\ol}[1]{{\overline{#1}}}
\newcommand{\av}[1]{\accentset{\circ}{\vec{#1}}}
\newcommand{\Ds}{D}
\newcommand\figTable[2]{\raisebox{-.5\height}{\includegraphics[scale=#1]{#2}}}

\newcommand{\transp}{\mathbf{T}}
\newenvironment{proof}{\noindent\ignorespaces\textit{Proof: }}{\hfill $\blacksquare$ \par\noindent\ignorespacesafterend\medskip}

\makeatletter
\DeclareFontFamily{U}{tipa}{}
\DeclareFontShape{U}{tipa}{m}{n}{<->tipa10}{}
\newcommand{\arc@char}{{\usefont{U}{tipa}{m}{n}\symbol{62}}}%

\newcommand{\arc}[1]{\mathpalette\arc@arc{#1}}

\newcommand{\arc@arc}[2]{%
  \sbox0{$\m@th#1#2$}%
  \vbox{
    \hbox{\resizebox{\wd0}{\height}{\arc@char}}
    \nointerlineskip
    \box0
  }%
}
\makeatother

  

\setbeamertemplate{bibliography entry title}{}
\setbeamertemplate{bibliography entry location}{}
\setbeamertemplate{bibliography entry note}{}

\tikzset{
  invisible/.style={opacity=0},
  visible on/.style={alt={#1{}{invisible}}},
  alt/.code args={<#1>#2#3}{%
    \alt<#1>{\pgfkeysalso{#2}}{\pgfkeysalso{#3}} % \pgfkeysalso doesn't change the path
  },
}

\title
  [Geometric Constraints and Variational Approaches to Image Analysis]
  {Geometric Constraints and Variational Approaches to Image Analysis}

\author[Daniel Martins Antunes]{  
  {Daniel Martins Antunes\textsuperscript{1}\\[1em]
  Supervised by: Jacques-Olivier Lachaud\textsuperscript{1} and Hugues Talbot\textsuperscript{2}}
}
  
  
\date
  {Le Bourget-du-Lac, 3 November 2020}


\institute
  {
	\textsuperscript{1}LAMA, Université Savoie Mont Blanc \\ 
	\textsuperscript{2}CentraleSupélec, Université Paris-Saclay
  }
 
\begin{document}
  \maketitle
  \captionsetup[subfigure]{labelformat=empty}

\begin{frame}
	{Presentation plan}

\begin{enumerate}
	{
	\item{Motivation}
	\begin{itemize}
		\item{Image analysis and geometric priors}
		\item{Elastica and completion property}		
		\item{State-of-the-art}							
	\end{itemize}}
	\vspace{1em}
	\item{Contribution}
	\begin{itemize}
		\item{Digital sets and convergent estimators}		
		\item{A combinatorial model for elastica}
		\item{A quadratic non-submodular formulation for elastica}	
		\item{Elastica minimization via graph-cuts}	
	\end{itemize}
	\vspace{1em}
	\item{Conclusion and perspectives}
\end{enumerate}

\end{frame}

\section{Motivation}

\begin{frame}
	{Motivation}	
	{Image analysis}

\begin{minipage}[t][0.5\textheight][t]{1\textwidth}
The problems we are interested in come from \emph{image analysis}.
\vspace{1em}

\only<1-4>{
\begin{center}
\begin{tabular}{ccc}
\highlight{2}{1,3-}{\textbf{Segmentation}} & 
\highlight{3}{1-2,4-}{\textbf{Denoising}} & 
\highlight{4}{1-3,5-}{\textbf{Inpainting}}
\end{tabular}
\end{center}}

\only<2>{
\center
\begin{tabular}{p{0.4\textwidth}c}
\includegraphics[scale=0.2]{figures/motivation/image-analysis/segmentation-cars.png} &
\includegraphics[scale=0.4]{figures/motivation/image-analysis/segmentation-crops.png}\\
\mycite{li2019gff} & \mycite{li2015object}
\end{tabular}}%
\only<3>{
\center
\begin{tabular}{cc}
\includegraphics[scale=0.44]{figures/motivation/image-analysis/denoising-airplane.png} &
\includegraphics[scale=0.22]{figures/motivation/image-analysis/denoising-mri.png} \\
\mycite{xu2018deep} & \mycite{jiang2018denoising}
\end{tabular}}%
\only<4>{
\center
\begin{tabular}{cc}
\includegraphics[scale=0.17]{figures/motivation/image-analysis/inpainting-man.png} &
\includegraphics[scale=0.24]{figures/motivation/image-analysis/inpainting-picture.png} \\
\mycite{yu2018generative} &  \mycite{masnou98inpainting}
\end{tabular}}%
\only<5->{
\begin{tabular}{p{0.6\textwidth}p{0.2\textwidth}}
\textbf{Segmentation:} $\mathcal{I}^{\star} = \argmin_{\mathcal{I}} E_{seg}(\mathcal{I},f_{\vec{I}})$ & \raisebox{-.5\height}{\includegraphics[scale=0.12]{figures/motivation/image-analysis/segmentation-stylised.png}}\\[2em]
\textbf{Denoising:} $f_{\widehat{\vec{I}}} = \argmin_f E_{den}(f,f_{\vec{\widetilde{I}}})$ & \raisebox{-.5\height}{\includegraphics[scale=0.12]{figures/motivation/image-analysis/denoising-stylised.png}}\\[2em]
\textbf{Inpainting:} $f_{\vec{\widehat{I}}} = \argmin_f E_{inp}(f,f_{\widetilde{\vec{I}}})$ & \raisebox{-.5\height}{\includegraphics[scale=0.12]{figures/motivation/image-analysis/inpainting-stylised.png}}
\end{tabular}}
\end{minipage}
%6
%
\begin{minipage}[t][0.27\textheight][t]{\textwidth}
\only<5->{We focused on \emph{variational approaches} to solve these problems.}

\only<6->{
Energies are defined according to assumptions made about the solution, e.g.,
\begin{itemize}
	\item{ \emph{Data fidelity}. The solution should not differ much from the input. }
	\item{ \emph{Spatial coherence}. Images are composed of regions with low variability in color. }
\end{itemize}
}
\end{minipage}

\end{frame}







\begin{frame}
{Motivation}
{Geometric priors}
The \emph{Mumford Shah}~(\mycite{mumford89}) is a model for segmentation and denoising.
%
%
\begin{align*}
	\min_{f,\highlight{4}{1-3,5-}{\mathcal{K}}} \alpha \int_{\Omega} \highlight{2}{1,3-}{\norm{ f_{\vec{I}} - f}^2}dx + \beta \int_{\Omega \setminus \highlight{4}{1-3,5-}{\mathcal{K}}} \highlight{3}{1,2,4-}{\norm{ \nabla f}^2} dx + \lambda Per(\highlight{4}{1-3,5-}{\mathcal{K}}).
\end{align*}
%
%
\onslide<5->{
The \emph{ROF}~(\mycite{rudin92}) model uses \emph{total variation} for image denoising
%
%
\begin{align*}
	\min_{f} \alpha \int_{\Omega} \norm{ f_{\vec{I}} - f}^2dx + \beta \int_{\Omega} \highlight{6}{1-5,7-}{\norm{ \nabla f }}dx.
\end{align*}}
%
%
\onslide<7->{
\begin{itemize}
	\item{A measure of length is present in both models.}
	\item{\emph{Geometric priors} as length, area or curvature are useful due to its flexibility and effect predictability.}
\end{itemize}}
%
%
\vspace{1em}
\onslide<8->{
In this thesis, we are interested in the combined use of \emph{length} and \emph{squared curvature} as geometric priors.}
\end{frame}

\begin{frame}
{Motivation}
{Completion property}
\begin{minipage}[t][0.5\textheight][t]{\textwidth}
\only<1->{
\center
$\min_{ \Omega \in \{\Omega_{c}, \Omega_{d} \} } Perimeter(\Omega).$
}
\only<1>{
\includegraphics[scale=0.25]{figures/motivation/completion/perimeter-0.png}
}
\only<2>{
\includegraphics[scale=0.25]{figures/motivation/completion/perimeter-1.png}
}
\only<3>{
\includegraphics[scale=0.25]{figures/motivation/completion/perimeter-2.png}
}
\only<4->{
\includegraphics[scale=0.25]{figures/motivation/completion/perimeter-3.png}
}
\end{minipage}
\begin{minipage}[t][0.5\textheight][t]{\textwidth}
\only<5->{
\center
$\min_{ \Omega \in \{\Omega_{c}, \Omega_{d} \} } Perimeter(\Omega) + Squared\;Curvature(\partial \Omega).$}
\only<5>{
\includegraphics[scale=0.25]{figures/motivation/completion/elastica-0.png}
}
\only<6>{
\includegraphics[scale=0.25]{figures/motivation/completion/elastica-1.png}
}
\only<7>{
\includegraphics[scale=0.25]{figures/motivation/completion/elastica-2.png}
}
\only<8->{
\includegraphics[scale=0.25]{figures/motivation/completion/elastica-3.png}
}
\end{minipage}
\end{frame}

\begin{frame}
{Motivation}
{Completion property}
\begin{minipage}[t][0.5\textheight][t]{\textwidth}
\only<1->{
\center
$\min_{ \Omega \in \{\Omega_{c}, \Omega_{d} \} } Perimeter(\Omega) + Squared\;Curvature(\partial \Omega).$}
\only<1>{
\includegraphics[scale=0.22]{figures/motivation/completion/elastica-g8a1-0.png}
}
\only<2>{
\includegraphics[scale=0.22]{figures/motivation/completion/elastica-g8a1-1.png}
}
\only<3>{
\includegraphics[scale=0.22]{figures/motivation/completion/elastica-g8a1-2.png}
}
\only<4>{
\includegraphics[scale=0.22]{figures/motivation/completion/elastica-g8a1-3.png}
}
\only<5->{
\includegraphics[scale=0.22]{figures/motivation/completion/elastica-g8a1-4.png}
}
\end{minipage}
\begin{minipage}[t][0.5\textheight][t]{\textwidth}
\only<6-9>{
\center
$\min_{ \Omega \in \{\Omega_{c}, \Omega_{d} \} } \frac{1}{2}Perimeter(\Omega) + Squared\;Curvature(\partial Omega).$}
\only<10->{
\center
\color{highlightcolor} $\min_{ \Omega \in \{\Omega_{c}, \Omega_{d} \} } \int_{\partial \Omega}{ \alpha + \beta \kappa ^2ds}. \quad - \quad \text{The elastica energy}$}
\only<6>{
\includegraphics[scale=0.22]{figures/motivation/completion/elastica-g8a05-0.png}
}
\only<7>{
\includegraphics[scale=0.22]{figures/motivation/completion/elastica-g8a05-1.png}
}
\only<8>{
\includegraphics[scale=0.22]{figures/motivation/completion/elastica-g8a05-2.png}
}
\only<9->{
\includegraphics[scale=0.22]{figures/motivation/completion/elastica-g8a05-3.png}
}
\onslide<11>{
	\begin{figure}
	\begin{tikzpicture}[overlay, remember picture] 
	\node at (current page.center) 
	    [
	    anchor=east,
	    xshift=-10mm,
	    yshift=0mm
	    ] 
	{
	
	\includegraphics[scale=0.16]{figures/motivation/completion/angiogram.jpg}
		
	};
	\node at (current page.center) 
	    [
	    anchor=west,
	    xshift=-10mm,
	    yshift=0mm
	    ] 
	{
	
	\includegraphics[scale=0.08]{figures/motivation/completion/paris-satellite-road.jpg}
		
	};	
	\end{tikzpicture}	
	\end{figure}	
	}	
\end{minipage}
\end{frame}

\begin{frame}
{Motivation}
{State-of-the-art}
\small
\textbf{Continuous setting}: Define the energy over the whole domain and minimize the elastica with respect the level-curves~(\mycite{chan02elasticainpainting}).
%
\begin{align*}
\int_{\Omega}{ \left(\alpha + \beta \nabla \cdot \left(\frac{\nabla f_{\vec{I}}}{\norm{\nabla f_{\vec{I}} }}\right) ^2 \right)\norm{\nabla f_{\vec{I}} }d\Omega}.
\end{align*}
%
\pause
\begin{itemize}
\item{Numerical instability: Fourth-order Euler-Lagrange equation.}
\item{Susceptible to bad local minimum.}
\end{itemize}
%
\pause
\vspace{0.5em}
\textbf{Discrete setting}:
\vspace{-1em}
\setlength\tabcolsep{3pt}
\begin{center}
\renewcommand{\arraystretch}{0.5}
\begin{tabular}{p{0.35\textwidth}p{0.65\textwidth}}
T-junctions matching & \multirow{2}{0.65\textwidth}{Fast algorithm, but limited to absolute value of curvature (polygonal solutions) and inpainting application.} \\
\mycite{masnou98inpainting}\pause &\\[2.75em]
Linear programming & \multirow{2}{0.65\textwidth}{Global optimization, but prohibitive running times even for small (thus unprecise) neighborhoods.} \\
\mycite{schoenemann09linear} & \\[2.75em]
Triple cliques & \multirow{2}{0.65\textwidth}{Global optimization, non-submodular energy. Limited precision due combinatorial explosion.} \\
\mycite{nieuwenhuis14efficient}
\end{tabular}
\end{center}
\end{frame}

\begin{frame}
{Motivation}
{Goals}

Models based on the minimization of the elastica energy

\center
\begin{tabular}{lcc|c|}
& Continuous & Discrete & \textbf{Digital} \\
\hline
Numerical instability & \negative{Yes} & \positive{No} & \positive{No} \\
Suitable for digital sets & \negative{No} & \negative{No} & \positive{Yes} \\
Rounding issues & \negative{Yes} & \positive{No} & \positive{No} \\
Global formulation & \positive{Yes} & \positive{Yes} & \negative{No} \\
Contour completion & \negative{Partial} & \negative{Partial} & \positive{Extended} \\
Global optimum (Free elastica) & \negative{-} & \negative{-} & \positive{Yes}
\end{tabular}

%\vspace{2em}
%
%\begin{itemize}
%\item{Can we define an elastica-based model for image analysis using multigrid convergent estimators? \only<4>{\positive{Yes!}}} \pause
%\item{Can we recover the completion property of elastica? \only<4>{\positive{Yes!}}} \pause
%\item{Can we escape bad local minima? \only<4>{\positive{Yes!}}}
%\end{itemize}

\end{frame}


\section{Digital images and convergent estimators}

\begin{frame}
\huge
\center
Digital images and convergent estimators
\vspace{2em}

\begin{minipage}{0.7\textwidth}
\normalsize
\center
\begin{itemize}
\item{Digital grid particularities and restrictions}
\item{Multigrid convergence of geometric estimators}
\end{itemize}
\end{minipage}

\end{frame}

\begin{frame}
{Digital images and convergent estimators}
{Digital set peculiarities}

\begin{minipage}[t][0.35\textheight][t]{1\textwidth}

Where do we think we can do better?

\only<2->{
\begin{itemize}
\item{Most of models neglect the digital character of digital images and ignore the fact that geometric measurements (mainly those local as tangent and curvature) in such objects should be done with a definition of \emph{convergence} that is specific for digital shapes.}
\end{itemize}
\vspace{1em}

}
\end{minipage}
%
%
\begin{minipage}[t][0.65\textheight][t]{1\textwidth}
\only<3>{
\textbf{Exact sampling x digitization}
\center
\begin{tabular}{ccc}
\includegraphics[scale=0.45]{figures/motivation/exact-sampling/sampling-0.png}&
\includegraphics[scale=0.45]{figures/motivation/exact-sampling/sampling-1.png}&
\includegraphics[scale=0.22]{figures/motivation/exact-sampling/digital-ball-perimeter.png}
\end{tabular}}%
\only<4>{
\textbf{Digitization ambiguity}

\begin{center}
\includegraphics[scale=1]{figures/motivation/exact-sampling/ambiguity.png}
\end{center}}
\end{minipage}
\end{frame}

\begin{frame}
{Digital images and convergent estimators}
{Multigrid convergent estimators}

\begin{definition}[Multigrid convergence]
	Let $\mathcal{X}$ be a family of shapes in $\mathbb{R}^n$ and $u$ a geometric quantity that is defined for every shape $X \in \mathcal{X}$. Further, let $D_h(X)$ denote the digitization of $X$ with grid step $h$.%
%
\vspace{1em}	
%	
	 The estimator $\hat{u}$ is multigrid convergent for $\mathcal{X}$ if and only if, for any $X \in \mathcal{X}$ there exists $h_X > 0$ such that for every $0< h < h_X$
	
	\begin{align*}
		| \hat{u}(D_h(X)) - u(X) | \leq \tau(h), \quad \text{with } \lim_{h\rightarrow 0}{\tau(h)} = 0
	\end{align*}	
\end{definition}
%
\pause
%
Multigrid convergent estimator of area
\begin{align*}
	\widehat{Area(X)} = h^2|D_h(X)|
\end{align*}
%
\end{frame}

\begin{frame}
{Motivation}
{Multigrid convergent estimators}

\footnotesize

\center
Disk of radius $5 (Area\approx78.54)$ 

\center
\begin{tabular}{ccc}
\includegraphics[scale=0.4]{figures/motivation/digital-geometric-estimators/multigrid/h1.png} &
\includegraphics[scale=0.4]{figures/motivation/digital-geometric-estimators/multigrid/h05.png} &
\includegraphics[scale=0.4]{figures/motivation/digital-geometric-estimators/multigrid/h025.png} \\
$h=1.0\; \widehat{A}=81$ & $h=\frac{1}{2}\; \hat{A}=79.25$ & $h=\frac{1}{4}\; \hat{A}=78.56$\\[1em]
\includegraphics[scale=0.4]{figures/motivation/digital-geometric-estimators/multigrid/h00625.png} &
\includegraphics[scale=0.4]{figures/motivation/digital-geometric-estimators/multigrid/h003125.png} &
\includegraphics[scale=0.4]{figures/motivation/digital-geometric-estimators/multigrid/h003125.png} \\
$h=\frac{1}{16}\; \hat{A}=78.44$ & $h=\frac{1}{32}\; \hat{A}=78.5$ & $h=\frac{1}{64}\; \hat{A}=78.53$
\end{tabular}

\end{frame}

\begin{frame}
	{Digital images and convergent estimators}	
	{Multigrid convergent estimators}	
%
	\begin{itemize}
		\onslide<1->{\item{Minimum Length Polygon (MLP)~\mycite{sloboda98approximation}}
		\begin{itemize}
			\item{Proved multigrid convergent for piecewise $3$-smooth convex shapes.}
		\end{itemize}}
		\onslide<3->{\vspace{2em}
		\item{Integral Invariant (II)~\mycite{coeurjolly13integral}}
		\begin{itemize}
			\item{Proved multigrid convergent for $C^2$ convex shapes with bounded curvature.}
		\end{itemize}}		
	\end{itemize}
	
	\onslide<2>{
	\begin{figure}
	\begin{tikzpicture}[overlay, remember picture] 
	\node at (current page.center) 
	    [
	    anchor=center,
	    xshift=0mm,
	    yshift=0mm
	    ] 
	{
	
	\includegraphics[scale=1.0]{figures/motivation/digital-geometric-estimators/mlp.png}
		
	};
	\end{tikzpicture}	
	\end{figure}	
	}
	
	\onslide<4->{
	\begin{figure}
	\begin{tikzpicture}[overlay, remember picture] 
	\node at (current page.center) 
	    [
	    anchor=center,
	    xshift=0mm,
	    yshift=0mm
	    ] 
	{
	\only<4>{
	\includegraphics[scale=0.5]{figures/motivation/digital-geometric-estimators/ii/zoom/fr3-zoom.png}}%
	\only<5>{
	\includegraphics[scale=0.5]{figures/motivation/digital-geometric-estimators/ii/zoom/fr5-zoom.png}}%
		
	};
	\end{tikzpicture}	
	\end{figure}	
	}	

	\vspace{1.5em}

	\onslide<4->{
	\begin{align*}
		\hat{\kappa}(p) = \frac{3}{r^3}\left( \frac{\pi r^2}{2} - | B_r(p) \cap X | \right )
	\end{align*}}		
	
\end{frame}

\begin{frame}
	{Digital images and convergent estimators}	
	{Conclusion}	

	\begin{itemize}
		\item{Digital sets are ambiguous and are constrained to the digital grid.}\\[1em]
		\item{The multigrid convergence is an adapted definition of convergence for geometric estimation on digital sets.}\\[2em]\pause
		\item[]{\textbf{Can we construct optimization models using multigrid convergent estimators?}}
	\end{itemize}

\end{frame}

\section{Combinatorial Elastica}

\begin{frame}
\begin{center}
\huge
A combinatorial model for elastica
\end{center}
\end{frame}

\begin{frame}
	{Combinatorial Elastica}	
	{Digital elastica}
%	
		\begin{definition}[Digital elastica energy]
		Let $\hat{s}$ and $\hat{\kappa}$ multigrid convergent estimators of local length and curvature. The digital elastica energy of a digital shape $S \subset \Omega \subset \mathbb{Z}^2$ is defined as
%		
		\begin{align*}
			\hat{E}(S) = \sum_{e \in \partial_h(S)}{\hat{s}(e) \Big(\: \alpha + \beta \hat{\kappa}^2(e) \: \Big). }
		\end{align*}
	\end{definition}
%	
	\begin{itemize}
		\item<2->{\emph{Local search}: set a local neighborhood $\mathcal{N}(S)$ of $S$ and pick the shape $X^{\star} \in \mathcal{N}(S)$ among those of minimum digital elastica value.}
	\end{itemize}
\end{frame}

\begin{frame}
	{Combinatorial Elastica}	
	{Local search algorithm}

\begin{algorithm}[H]
 \SetKwData{It}{i}
 \SetKwData{MIt}{maxIt}
 \SetKwData{Tol}{tolerance}
 \SetKwData{Delta}{delta}
 \SetKwData{Best}{best} 
 \SetKwInOut{Input}{input}\SetKwInOut{Output}{output}
 
 \Input{A digital set $S$; coefficient $\alpha, \beta$; the maximum number of iterations \MIt; and a stop condition \Tol}
 \BlankLine
 \Delta $\longleftarrow$ \Tol+1\;
 $i \longleftarrow 0$\;
 $S^{(0)} \longleftarrow S$\;
 $X^\star \longleftarrow S$\;
 \While{ \color{black} \It $<$ \MIt \bf{and} \Delta $>$ \Tol  }{
  	\For{$\color{black} X \in \mathcal{N}(S^{(i)}) $}
	{
		\If{\color{black} $\hat{E}(X)$ $<$ $\hat{E}(X^\star)$ }
		{
			$X^\star \longleftarrow X$
		}
	}
	\It $\longleftarrow$ \It $+1$\;
	$S^{(i)} \longleftarrow X^\star$\;
	\Delta $\longleftarrow$ $\hat{E}(S^{(i-1)}) - \hat{E}(S^{(i)})$\;	
 }
\end{algorithm}
\end{frame}

\begin{frame}
	{Combinatorial Elastica}	
	{Neighborhood of shapes}
	
\begin{minipage}{0.49\textwidth}
\center
\includegraphics[scale=0.25]{figures/combinatorial-elastica/main-inner.png}	
\end{minipage}
\begin{minipage}{0.49\textwidth}
\includegraphics[scale=0.25]{figures/combinatorial-elastica/main-outer.png}	
\end{minipage}	
\end{frame}

\begin{frame}
	{Combinatorial Elastica}	
	{Shape evolution}

\begin{center}
$\alpha=0.01, \beta=1$
\end{center}
\begin{minipage}{0.49\textwidth}
\center
\includegraphics[scale=0.15]{figures/combinatorial-elastica/flow/ii/elastica/len_pen_0.01000/jonctions_1/curve_segs_4/best/gs_0.25000/triangle.png}\\[1em]
\includegraphics[scale=0.18]{figures/combinatorial-elastica/flow/ii/elastica/len_pen_0.01000/jonctions_1/curve_segs_4/best/gs_0.25000/flower.png}
\end{minipage}	
\begin{minipage}{0.49\textwidth}
\center
\includegraphics[scale=0.15]{figures/combinatorial-elastica/flow/ii/elastica/len_pen_0.01000/jonctions_1/curve_segs_4/best/gs_0.25000/square.png}\\[1em]
\includegraphics[scale=0.18]{figures/combinatorial-elastica/flow/ii/elastica/len_pen_0.01000/jonctions_1/curve_segs_4/best/gs_0.25000/bean.png}
\end{minipage}

\end{frame}

\begin{frame}
	{Combinatorial Elastica}	
	{Energy evolution}

\begin{minipage}{0.49\textwidth}
\center
\includegraphics[scale=0.3]{figures/combinatorial-elastica/flow/ii/elastica/len_pen_0.01000/jonctions_1/curve_segs_4/best/gs_0.25000/summary-ii5.png}
\end{minipage}
\begin{minipage}{0.49\textwidth}
\begin{align*}
	\min E(X) &= \int_{\partial X}{ \alpha + \beta \kappa^2 ds}\\
	 &= 4\pi \beta \frac{1}{r} = 4\pi \beta \left(\frac{\alpha}{\beta}\right)^{1/2}
\end{align*}
, where $\frac{\partial }{\partial r} 2\pi(\alpha r + \frac{\beta}{r}) = 0$\\ 

%
For $\alpha=0.01,\; \beta=1$
%
\begin{align*}
	\min E(X) \approx 1.2566
\end{align*}
\end{minipage}
	
\end{frame}

\begin{frame}
	{Combinatorial Elastica}
	{Radius and grid resolution}
	
\begin{minipage}{0.49\textwidth}
\center
\includegraphics[scale=0.25]{figures/combinatorial-elastica/flow/ii/elastica/len_pen_0.01000/jonctions_1/curve_segs_4/best/gs_0.25000/triangle-bars.png}\\[0.6em]
\includegraphics[scale=0.25]{figures/combinatorial-elastica/flow/ii/elastica/len_pen_0.01000/jonctions_1/curve_segs_4/best/gs_0.25000/flower-bars.png}
\end{minipage}	
\begin{minipage}{0.49\textwidth}
\center
\includegraphics[scale=0.25]{figures/combinatorial-elastica/flow/ii/elastica/len_pen_0.01000/jonctions_1/curve_segs_4/best/gs_0.25000/square-bars.png}\\[0.6em]
\includegraphics[scale=0.25]{figures/combinatorial-elastica/flow/ii/elastica/len_pen_0.01000/jonctions_1/curve_segs_4/best/gs_0.25000/bean-bars.png}
\end{minipage}
\end{frame}

\begin{frame}
	{Combinatorial Elastica}
	{Other experiments}
	
\begin{minipage}{0.49\textwidth}
\center
\includegraphics[scale=0.25]{figures/combinatorial-elastica/other-experiments/ii/elastica/len_pen_0.001000/jonctions_1/curve_segs_4/best/gs_0.25000/flower.png}
\end{minipage}
\begin{minipage}{0.49\textwidth}
\center
\includegraphics[scale=0.25]{figures/combinatorial-elastica/other-experiments/ii/elastica/len_pen_0.001000/jonctions_1/curve_segs_4/best/gs_0.25000/curve.png}
\end{minipage}
\end{frame}

\begin{frame}
{Combinatorial Elastica}
{Running time}
\begin{center}
\captionsetup{type=table}
\scriptsize
\begin{tabular}{|l|c|c|c|c|c|c|}
\hline
& \multicolumn{2}{c|}{$h=1.0$} & \multicolumn{2}{c|}{$h=0.5$} & \multicolumn{2}{c|}{$h=0.25$}\\
\hline
& Pixels & Time & Pixels & Time & Pixels & Time\\
\hline
Triangle & 521 & 2s (0.07s/it)  & 2080 & 43s (0.81s/it) & 8315 & 532s(4.8s/it)\\
Square & 841 & 0.9s (0.09s/it) & 3249 & 8s (0.3s/it) & 12769 & 102s (2s/it)\\
Flower & 1641 & 13s (0.24s/it) & 6577 & 209s (1.68s/it) & 26321 & 3534s (12.3s/it)\\
Bean  & 1574 & 7s (0.16s/it) & 6278 & 88s (1.08s/it) & 25130 & 1131s (6.4s/it)\\
Ellipse  & 626 & 1s (0.14s/it) & 2506 & 16s (0.44s/it) & 10038 & 286s (3.1s/it)\\
\hline
\end{tabular}
\caption{\textbf{Running time of LocalSearch.} The running times for the free elastica problem are displayed. Notice that even having a similar number of pixels, the square (bean) shape evolves much faster than the triangle (flower).}
\end{center}
\end{frame}

\begin{frame}
{Combinatorial Elastica}
{Conclusion}

\begin{itemize}
\item{Multigrid convergent estimators are suitable for elastica minimization}\pause
\item{A simple neighborhood is sufficient to escape bad local minima. Some solutions very close to global optimum.}\pause
\item{Too slow. It cannot be used in practice.}
\end{itemize}
\end{frame}
\section{Non-submodular elastica}

\begin{frame}
\begin{center}
\huge
A quadratic non-submodular formulation for elastica
\end{center}
\end{frame}

\begin{frame}
	{Non-submodular elastica}	
	{Difficulties with a global model}
\begin{minipage}[t][0.6\textheight][t]{1\textwidth}
\begin{minipage}{0.35\textwidth}
\only<1>{
\includegraphics[scale=0.5]{figures/non-submodular-elastica/global/issues-2.png}
}
\only<2>{
\includegraphics[scale=0.5]{figures/non-submodular-elastica/global/issues-3.png}
}
\only<3>{
\includegraphics[scale=0.5]{figures/non-submodular-elastica/global/issues-4.png}
}
\only<4->{
\includegraphics[scale=1]{figures/non-submodular-elastica/global/topological-constraints.png}
}
\end{minipage}
%
%
\begin{minipage}{0.64\textwidth}
\onslide<2->{
\begin{itemize}
	\item{Center of the estimation disk}
	\onslide<3->{\item{Pixel counting and estimation of curvature squared}}
	\onslide<4->{\item{Topological constraints}}	
	\onslide<5->{\item{Third order non-convex binary}}		
	\onslide<6->{\item{Level 1 linearization: non semi-definite positive matrix}}
	\onslide<7->{\item{Level 2 linearization: $O(m^3)$ variables}}
\end{itemize}}
\end{minipage}	
\end{minipage}
%
%
\begin{minipage}[t][0.25\textheight][t]{1\textwidth}
\only<2>{
\begin{align*}
	\sum_{\ell_i \in \mathcal{L}}{ \vec{y}_i \left(\; \alpha + \beta \hat{\kappa}_{r}^2(D,\ell_i) \; \right)}\\\nonumber
\end{align*}}
\only<3>{
\begin{align*}
&\sum_{\ell_i \in \mathcal{L}}{ \vec{y}_i \left(\; \alpha + \frac{9}{r^6}\beta \big(c^2 - 2c\vec{A}_i^T\vec{x} + \vec{x}^T\vec{A}_i\vec{A}_i^T\vec{x}\big)\right)}\\
&\text{subject to} \quad \vec{x} \in \{0,1\}^m, \vec{y} \in \{0,1\}^n.
\end{align*}}
\only<4->{
\begin{align*}
&\sum_{\ell_i \in \mathcal{L}}{ \vec{y}_i \left(\; \alpha + \frac{9}{r^6}\beta \big(c^2 - 2c\vec{A}_i^T\vec{x} + \vec{x}^T\vec{A}_i\vec{A}_i^T\vec{x}\big)\right)}\\
&\text{subject to} \quad \vec{x} \in \{0,1\}^m, \vec{y} \in \{0,1\}^n, T(\vec{x},\vec{y}).
\end{align*}}
\end{minipage}

\end{frame}

\begin{frame}
{Non-submodular elastica}
{Simplification}
\center
$\hat{\kappa}(p) = \frac{3}{r^3}\left( \frac{\pi r^2}{2} - | B_r(p) \cap X | \right )$
\begin{minipage}[t][0.5\textheight]{1\textwidth}
\center
\only<1>{
\includegraphics[scale=0.5]{figures/non-submodular-elastica/current-contour.png}}
\only<2>{
\includegraphics[scale=0.1]{figures/non-submodular-elastica/before-opt.png}\hspace{2em}
\includegraphics[scale=0.1]{figures/non-submodular-elastica/shape-opt-ball-after.png}}
\end{minipage}

\begin{itemize}
\item{Define an optimization band (yellow) as the inner contour of the shape, denoted $I$.}
\item{Set pixels such that the curvature estimation is reduced}
\end{itemize}
\end{frame}

\begin{frame}
{Non-submodular elastica}
{Simplification}
\center
$\hat{\kappa}(p) = \frac{3}{r^3}\left( \frac{\pi r^2}{2} - | B_r(p) \cap X | \right )$
\begin{minipage}[t][0.5\textheight]{1\textwidth}
\center
\includegraphics[scale=0.1]{figures/non-submodular-elastica/before-opt.png}\hspace{2em}
\only<1-2>{
\includegraphics[scale=0.1]{figures/non-submodular-elastica/shape-opt-ball-after.png}}
\only<3>{
\includegraphics[scale=0.1]{figures/non-submodular-elastica/shape-opt-after-inverted.png}}
\end{minipage}

\begin{itemize}
\item{Optimization identifies zones of shortage (convex) or abundance (concave) of pixels.}
\onslide<2->{
\item{$x=1 \rightarrow$ Zone of shortage of pixels (convex) $\rightarrow$ Estimator disk should be shifted towards the interior $\rightarrow$ This pixel does not belong to the next contour. }}
\end{itemize}
\end{frame}

\begin{frame}
{Non-submodular elastica}
{FlipFlow}
\begin{minipage}[t][0.6\textheight][t]{1\textwidth}
\footnotesize
\only<1-5>{
\begin{align*}
  E_{(\vec{\theta},m)}^{flip}( \highlight{2}{1,3-5}{ \Ds^{(k)},X^{(k)} } ) =& \sum_{ x_j \in X^{(k)}}{ \alpha s(x_j)} +  \sum_{ p \in \highlight{3}{1-2,4-}{R_m(\Ds^{(k)})}}{\beta \hat{\kappa}(p)^2}\\ 
  \onslide<4->{=& \sum_{ x_j \in X^{(k)}}{ \alpha \highlight{5}{1-4,6-}{s(x_j)}} \\
  +&\sum_{ \substack{p \in \\ R_m(\Ds^{(k)})}}{ 2c_1 \beta  \Big( { (1/2+ |F_{r}^{(k)}(p)|-c_2) \cdot \sum_{ \substack{ x_j \in \\ X_{r}^{(k)}(p)}}{x_j} } + \sum_{ \substack{j<l, \\ x_j,x_l \in \\ X_{r}^{(k)}(p) } }{x_jx_l} \Big) }}
\end{align*}}
\only<6->{
\begin{align*}
  E_{(\vec{\theta},m)}^{flip}( \highlight{2}{1,3-5}{ \Ds^{(k)},{\highlight{6}{7-}{1-X^{(k)}} }} ) =& \sum_{ x_j \in X^{(k)}}{ \alpha s(x_j)} +  \sum_{ p \in \highlight{3}{1-2,4-}{R_m(\Ds^{(k)})}}{\beta \hat{\kappa}(p)^2}\\ 
  \onslide<4->{=& \sum_{ x_j \in X^{(k)}}{ \alpha \highlight{5}{1-4,6-}{s(x_j)}} \\
  +&\sum_{ \substack{p \in \\ R_m(\Ds^{(k)})}}{ 2c_1 \beta  \Big( { (1/2+ |F_{r}^{(k)}(p)|-c_2) \cdot \sum_{ \substack{ x_j \in \\ X_{r}^{(k)}(p)}}{x_j} } + \sum_{ \substack{j<l, \\ x_j,x_l \in \\ X_{r}^{(k)}(p) } }{x_jx_l} \Big) }}
\end{align*}}
\end{minipage}
%
%
\begin{minipage}[t][0.39\textheight][t]{1\textwidth}
\only<2-3>{
\begin{center}
$\displaystyle
D \subset \Omega \subset \mathbb{Z}^2, \quad
X^{(k)} := \{ \; x_i \in \{0,1\} \; | \; p_i \in I^{(k)} \; \}$
\end{center}}
\only<3>{
\begin{center}
$\displaystyle
R_m(D) := \{ p \; | \; m-1 < d_D(p) \leq m \} \cup \{ p \; | \; -m+1 > d_D(p) \geq -m \}$
\end{center}}
\only<5>{
\begin{center}
\[
  s(x_j)=\sum_{q_i \in \mathcal{N}_4(p_j)}{ t(q_i) }, \quad \text{where } t(q_i) = \left\{\begin{array}{ll}
  (x_j-x_i)^2, & \text{if } q_i \in I^{(k)}\\
  (x_j-1)^2, & \text{if } q_i \in F^{(k)}\\
  (x_j-0)^2, & \text{otherwise. }
  \end{array}\right.
\]
\end{center}}
\only<6>{
\begin{center}
\[
  s(x_j)=\sum_{q_i \in \mathcal{N}_4(p_j)}{ t(q_i) }, \quad \text{where } t(q_i) = \left\{\begin{array}{ll}
  (x_j-x_i)^2, & \text{if } q_i \in I^{(k)}\\
  (x_j-{\color{blue}0})^2, & \text{if } q_i \in F^{(k)}\\
  (x_j-{\color{blue}1})^2, & \text{otherwise. }
  \end{array}\right.
\]
\end{center}}
\begin{minipage}{0.49\textwidth}
\scriptsize
\only<8->{\transparent{0.4}}
\only<7->{
\begin{center}
Shrink mode (convexities)
\[
\begin{array}{rl}
	a^{(k)} &\leftarrow \displaystyle \argmin_{X^{(k)}} E_{\vec{\Theta},m}^{flip}(D^{(k)},1-X^{(k)});\\[1.5em]
	D^{(k+1)} &\leftarrow F^{(k)} + a^{(k)}.
\end{array}
\]
\end{center}}
\end{minipage}
\begin{minipage}{0.49\textwidth}
\scriptsize
\only<8->{
\begin{center}
Expansion mode (concavities)
\[
\begin{array}{rl}
	a^{(k)} &\leftarrow 	\displaystyle \argmin_{\overline{X}^{(k)}} E_{\vec{\Theta},m}^{flip}(\overline{D}^{(k)},1-\overline{X}^{(k)});\\[1.5em]
	D^{(k+1)} &\leftarrow \overline{ \overline{F}^{(k)} + a^{(k)} }.
\end{array}
\]
\end{center}}
\end{minipage}
\end{minipage}

\end{frame}

\begin{frame}
{Non-submodular elastica}
{Evaluation on farther rings}

\begin{minipage}{0.49\textwidth}
\center
$r=3$\\
\includegraphics[scale=0.2]{figures/non-submodular-elastica/radius-effect/triangle-r3.png}\\[1em]
\includegraphics[scale=0.2]{figures/non-submodular-elastica/radius-effect/flower-r3.png}
\end{minipage}
\begin{minipage}{0.49\textwidth}
\center
$r=5$\\
\includegraphics[scale=0.2]{figures/non-submodular-elastica/radius-effect/triangle-r5.png}\\[1em]
\includegraphics[scale=0.2]{figures/non-submodular-elastica/radius-effect/flower-r5.png}
\end{minipage}

\end{frame}

\begin{frame}
{Non-submodular elastica}
{Evaluation on farther rings}
\begin{tabular}{cccc}
\multicolumn{4}{c}{$r=5$}\\
$m=1$ & $m=3$ & $m=4$ & $m=5$ \\
\includegraphics[scale=0.13]{figures/non-submodular-elastica/level-effect/triangle-l1.png}&
\includegraphics[scale=0.13]{figures/non-submodular-elastica/level-effect/triangle-l3.png}&
\includegraphics[scale=0.13]{figures/non-submodular-elastica/level-effect/triangle-l4.png}&
\includegraphics[scale=0.13]{figures/non-submodular-elastica/level-effect/triangle-l5.png}\\[2em]
\includegraphics[scale=0.13]{figures/non-submodular-elastica/level-effect/flower-l1.png}&
\includegraphics[scale=0.13]{figures/non-submodular-elastica/level-effect/flower-l3.png}&
\includegraphics[scale=0.13]{figures/non-submodular-elastica/level-effect/flower-l4.png}&
\includegraphics[scale=0.13]{figures/non-submodular-elastica/level-effect/flower-l5.png}
\end{tabular}
\end{frame}

\begin{frame}
{Non-submodular elastica}
{Contour correction}
\begin{tabular}{cc}
\includegraphics[scale=0.28]{figures/non-submodular-elastica/contour-correction/gc-seg-airplane.png}&
\includegraphics[scale=0.28]{figures/non-submodular-elastica/contour-correction/corrected-seg-airplane.png}\\[1em]
\includegraphics[scale=0.28]{figures/non-submodular-elastica/contour-correction/gc-seg-panther.png}&
\includegraphics[scale=0.28]{figures/non-submodular-elastica/contour-correction/corrected-seg-panther.png}
\end{tabular}
\end{frame}

\begin{frame}
{Non-submodular elastica}
{Unlabeled ratio}
\includegraphics[scale=0.28]{figures/non-submodular-elastica/level-effect/plot-unlabeled-triangle.png}\hspace{1em}
\includegraphics[scale=0.28]{figures/non-submodular-elastica/level-effect/plot-unlabeled-flower.png}
\begin{itemize}
{\only<3->{\transparent{0.4}}
\onslide<2->{\item{Given that we evaluate the curvature estimator on the initial contour, how to change the pixels in this contour to reduce the difference of inner and outer pixels?}}
}
\onslide<3->{\item{Given that the shape does not change, where the estimation disks should be centered in order to reduce the difference of inner and outer pixels?}}
\end{itemize}

\end{frame}

\begin{frame}
{Non-submodular elastica}
{Balance coefficient}
\begin{minipage}{0.5\textwidth}
\center
\includegraphics[scale=0.2]{figures/non-submodular-elastica/balance-coefficient-zero-level-set.png}
\end{minipage}
\begin{minipage}{0.49\textwidth}
\footnotesize
\begin{itemize}
\item{Balance coefficient}
\begin{align*}
u_r(D,p) &= \left( \frac{\pi r^2}{2} - |B_r(p) \cap D| \right)^2
\end{align*}
\item{White contour: contour of the shape}
\item{Pink contour: zero level set of the balance coefficient}
\end{itemize}
\end{minipage}
\end{frame}

\begin{frame}
{Non-submodular elastica}
{Conclusion}

\begin{itemize}
\item{Change pixels in the contour in order to reduce the difference between inner and outer pixels.}
\item{Quadratic non-submodular energy.}
\pause
\item{Farther rings: better results when disks are evaluated farther from the contour. }
\pause
\item{Pos-processing procedure: contour correction. }
\pause
\item{Unstable hypothesis: fixing the contour (dimension 1) is more sensitive than fixing the shape (dimension 2).}
\item{Make the shape evolve to the zero level set of its balance coefficient.}
\end{itemize}
\end{frame}


%\begin{align*}
%  s(x_{w(p)})=\sum_{q \in \mathcal{N}_4(p)}{ t(q) }, \quad \text{where } t(q) = \left\{\begin{array}{ll}
%  (x_{w(p)}-x_{w(q)})^2, & \text{if } q \in I^{(k)}\\
%  (x_{w(p)}-1)^2, & \text{if } q \in F^{(k)}\\
%  (x_{w(p)}-0)^2, & \text{otherwise. }
%  \end{array}\right.
%\end{align*}
\section{Elastica minimization via graph-cuts}

\begin{frame}
\center
\huge
Elastica minimization via graph-cuts

\vspace{2em}

\begin{minipage}{0.7\textwidth}
\normalsize
\begin{itemize}
\item{Balance coefficient to estabilize curvature estimation.}
\item{Set up a graph whose minimum cut approximates the zero level set of the balance coefficient.}
\item{GraphFlow algorithm. Up to $10$x faster than FlipFlow.}
\end{itemize}
\end{minipage}

\end{frame}

\begin{frame}
{Non-submodular elastica}
{Balance coefficient}
\begin{minipage}{0.5\textwidth}
\center
\includegraphics[scale=0.2]{figures/non-submodular-elastica/balance-coefficient-zero-level-set.png}
\end{minipage}
\begin{minipage}{0.49\textwidth}
\footnotesize
\begin{itemize}
\item{Balance coefficient}
\begin{align*}
u_r(D,p) &= \left( \frac{\pi r^2}{2} - |B_r(p) \cap D| \right)^2
\end{align*}
\item{White contour: contour of the shape}
\item{Pink contour: $\epsilon$-level set of the balance coefficient}
\end{itemize}
\end{minipage}
\end{frame}

\begin{frame}
{Elastica minimization via graph-cuts}
{Graph cut}
\center
\only<1>{
\includegraphics[scale=0.25]{figures/graphcut/cut-1.png}}%
\only<2>{
\includegraphics[scale=0.25]{figures/graphcut/cut-2.png}}%
\only<3>{
\includegraphics[scale=0.25]{figures/graphcut/cut-3.png}}%
\end{frame}


\begin{frame}
{Elastica minimization via graph-cuts}
{Building the graph}

\begin{minipage}{0.3\textwidth}
\only<1>{
\includegraphics[scale=0.5]{figures/graphcut/graph-model-1.png}}%
\only<2-4>{
\includegraphics[scale=0.5]{figures/graphcut/graph-model-2.png}}%
\only<5->{
\includegraphics[scale=0.5]{figures/graphcut/graph-model-3.png}}%
\end{minipage}
%
%
\begin{minipage}{0.69\textwidth}
\footnotesize
\begin{itemize}
\onslide<2->{\item{Optimization band
\begin{align*}
\only<2>{O_n(D) :=& \{ p \in D \; | \; -n \leq d_D(p) \leq n \} \\}
\only<3->{O(D) :=& \{ p \in D \; | \; -n \leq d_D(p) \leq n \} \\}
\onslide<4->{F(D) :=& D \setminus O(D)}
\end{align*}}}
\onslide<5->{\item{Graph $\mathcal{G}_D(\mathcal{V},\mathcal{E},c)$
\begin{align*}
\mathcal{V} &= \{ v_p \; | \; p \in O(D) \} \cup \{s,t\} \\
\highlight{7}{5,6,8-}{\mathcal{E}} &= \highlight{7}{5,6,8-}{\{ \{v_p,v_q\} \; | \; p,q \in O(D) \text{ and } q \in \mathcal{N}_4(p) \}} \cup \highlight{6}{5,7-}{\mathcal{E}_{st}} \\
\highlight{6}{5,7-}{\mathcal{E}_{st}} &= \highlight{6}{5,7-}{\{ (s,v_p), (v_p,t) \; | \; p \in O(D) \}}
\end{align*}}}
\onslide<8->{\item{Edge's weight
\begin{center}
\begin{tabular}{|c|c|}
\hline
\textbf{edge} $e$ & $\mathbf{c(e)}$\\
\hline
$\{v_p, v_q\}$ & $ \frac{1}{2}\left( u_r(D,p) + u_r(D,q)\right) $\\
\hline
$(s,v_p)$ & $M$\\
\hline
$(v_p, t)$ & $M$\\
\hline
\end{tabular}
\end{center}
}}
\onslide<9->{\item{Digital shape update
\begin{align*}
D^{(k+1)} &= F(D^{(k)}) + S^{(k)}
\end{align*}}}
\end{itemize}
\end{minipage}
\end{frame}
%
%
%
\begin{frame}
{Elastica minimization via graph-cuts}
{Shape evolution}

\begin{center}
$\alpha=1/8^2, \beta=1.$\\[1em]

\begin{tabular}{cc}
\includegraphics[scale=0.1]{figures/graphcut/no-neighborhood-flow-always-evolve/0.015625/triangle.png}\hspace{3em} &
\includegraphics[scale=0.08]{figures/graphcut/no-neighborhood-flow-always-evolve/0.015625/square.png}\\[1em]
\includegraphics[scale=0.12]{figures/graphcut/no-neighborhood-flow-always-evolve/0.015625/flower.png}\hspace{3em} &
\includegraphics[scale=0.12]{figures/graphcut/no-neighborhood-flow-always-evolve/0.015625/bean.png}
\end{tabular}
\end{center}

\onslide<2->{
\begin{itemize}
\item{What if we stop the evolution when elastica increases?}
\end{itemize}}

\end{frame}

\begin{frame}
{Elastica minimization via graph-cuts}
{Shape evolution}

\begin{center}
Stop if elastica increases $(\alpha=1/8^2,\beta=1)$\\[1em]

\begin{tabular}{cc}
\includegraphics[scale=0.1]{figures/graphcut/no-neighborhood-flow-always-improve/0.015625/triangle.png}\hspace{3em} &
\includegraphics[scale=0.08]{figures/graphcut/no-neighborhood-flow-always-improve/0.015625/square.png}\\[2em]
\includegraphics[scale=0.12]{figures/graphcut/no-neighborhood-flow-always-improve/0.015625/flower.png}\hspace{3em} &
\includegraphics[scale=0.12]{figures/graphcut/no-neighborhood-flow-always-improve/0.015625/bean.png}
\end{tabular}

\end{center}




\end{frame}

\begin{frame}
{Elastica minimization via graph-cuts}
{Shape evolution}

\begin{center}
Stop if elastica increases $(\alpha=1/22^2,\beta=1)$\\[1em]

\begin{tabular}{cc}
\includegraphics[scale=0.12]{figures/graphcut/no-neighborhood-flow-always-improve/0.0020661157/triangle.png}\hspace{3em} &
\includegraphics[scale=0.12]{figures/graphcut/no-neighborhood-flow-always-improve/0.0020661157/square.png}\\[2em]
\includegraphics[scale=0.12]{figures/graphcut/no-neighborhood-flow-always-improve/0.0020661157/flower.png}\hspace{3em} &
\includegraphics[scale=0.12]{figures/graphcut/no-neighborhood-flow-always-improve/0.0020661157/bean.png}
\end{tabular}

\end{center}




\end{frame}

\begin{frame}
{Elastica minimization via graph-cuts}
{The $a$-probe set}

\begin{definition}[$a$-probe set]
	Let $\Ds \subset \Omega \subset \mathbb{Z}^2$ a digital set and $a$ a natural number. The $a$-probe set of $\Ds$ is defined as
	\begin{align*}
		\mathcal{P}_a(\Ds) &= \Ds \cup \bigcup_{a' < a}{\Ds^{+a'} \cup \Ds^{-a'}},
	\end{align*}
	where $\Ds^{+a}$($\Ds^{-a}$) denotes a dilation(erosion) by a disk of radius $a$.
\end{definition}

\textbf{Candidate selection}
\[\begin{array}{l}
	sol(D^{(k)}) \longleftarrow \bigcup_{D' \in \mathcal{P}_a(D^{(k)})} \Big\{ F^{(k)} + S  \; | \; mincut(S,\mathcal{G}_{D'}) \Big\} 	
\end{array}
\]

\textbf{Candidate validation}
\[\begin{array}{l}
\Ds^{(k+1)} \longleftarrow \displaystyle \argmin_{ D' \in sol(D^{(k)}) }{ \hat{E}_{\vec{\theta}}(D')}
\end{array}
\]

\end{frame}

\begin{frame}
{Elastica minimization via graph-cuts}
{Shape evolution with $a$-probe set}

\center
\only<1>{Stop if elastica increases $(\alpha=1/22^2,\beta=1)$ \\[1em]}
\only<2>{Always update $(\alpha=1/22^2,\beta=1)$ \\[1em]}
\only<1>{
\begin{tabular}{cc}
\includegraphics[scale=0.12]{figures/graphcut/with-neighborhood-flow-always-improve/triangle.png}\hspace{3em} &
\includegraphics[scale=0.12]{figures/graphcut/with-neighborhood-flow-always-improve/square.png}\\[2em]
\includegraphics[scale=0.12]{figures/graphcut/with-neighborhood-flow-always-improve/flower.png}\hspace{3em} &
\includegraphics[scale=0.12]{figures/graphcut/with-neighborhood-flow-always-improve/bean.png}
\end{tabular}}
%
%
\only<2>{
\begin{tabular}{cc}
\includegraphics[scale=0.12]{figures/graphcut/with-neighborhood-flow/radius_16/triangle.png}\hspace{3em} &
\includegraphics[scale=0.12]{figures/graphcut/with-neighborhood-flow/radius_16/square.png}\\[2em]
\includegraphics[scale=0.12]{figures/graphcut/with-neighborhood-flow/radius_16/flower.png}\hspace{3em} &
\includegraphics[scale=0.12]{figures/graphcut/with-neighborhood-flow/radius_16/bean.png}
\end{tabular}}
\end{frame}

\begin{frame}
{Elastica minimization via graph-cuts}
{Shape evolution with $a$-probe set}

\begin{minipage}{0.25\textwidth}
\center
\includegraphics[scale=0.06]{figures/graphcut/with-neighborhood-flow/radius_16/triangle.png}\\[1em]
\includegraphics[scale=0.06]{figures/graphcut/with-neighborhood-flow/radius_16/square.png}\\[1em]
\includegraphics[scale=0.06]{figures/graphcut/with-neighborhood-flow/radius_16/flower.png}\\[1em]
\includegraphics[scale=0.06]{figures/graphcut/with-neighborhood-flow/radius_16/bean.png}
\end{minipage}%
%
%
\begin{minipage}{0.74\textwidth}
\center
\includegraphics[scale=0.18]{figures/graphcut/with-neighborhood-flow/plots/elastica.png}\\
\includegraphics[scale=0.18]{figures/graphcut/with-neighborhood-flow/plots/bars.png}
\end{minipage}
\end{frame}

\begin{frame}
{Elastica minimization via graph-cuts}
{Contour correction}

\begin{minipage}{0.5\textwidth}
\center
\only<1>{
\includegraphics[scale=0.45]{figures/graphcut/contour-correction/cat/gc-seg.png}

Initial segmentation

}%
\only<2>{
\includegraphics[scale=0.45]{figures/graphcut/contour-correction/vase/gc-seg.png}

Initial segmentation

}%
\only<3,4>{
\includegraphics[scale=0.4]{figures/graphcut/contour-correction/coala/gc-seg.png}

Initial segmentation

}%
\end{minipage}%
\begin{minipage}{0.5\textwidth}
\center
\only<1>{
\includegraphics[scale=0.45]{figures/graphcut/contour-correction/cat/corrected-seg.png}

$0.825s$ ($3$ it)

}%
\only<2>{\includegraphics[scale=0.45]{figures/graphcut/contour-correction/vase/corrected-seg.png}

$0.746s$ ($3$ it)

}%
\only<3>{\includegraphics[scale=0.4]{figures/graphcut/contour-correction/coala/5it.png}

$1.1s$ ($3$ it)

}%
\only<4>{\includegraphics[scale=0.4]{figures/graphcut/contour-correction/coala/30it.png}

$10s$ ($30$ it)

}%
\end{minipage}%

\end{frame}

\begin{frame}
{Elastica minimization via graph-cuts}
{Contour completion}

\begin{minipage}[t][0.47\textheight]{1\textwidth}
\center
\includegraphics[scale=0.28]{figures/graphcut/contour-completion/green-snake/gc-seg.png}

Initial segmentation

\end{minipage}
\vspace{1em}
\begin{minipage}[t][0.47\textheight]{1\textwidth}
\center
\includegraphics[scale=0.28]{figures/graphcut/contour-completion/green-snake/corrected-seg.png}

$17s$ ($62$ it)

\end{minipage}


\end{frame}

\section{Conclusion}

\begin{frame}
{Conclusion}
{Summary of models}

\onslide<1->{
\begin{table}[H]
\footnotesize
\centering
\begin{tabular}{r|ccccc}
Model & Implementation & RT & Free & Constrained & Image\\
\hline
LocalSearch (LS) & medium & slow & yes(opt) & yes & no \\
FlipFlow (FF) & hard & acceptable & yes & no & yes \\
BalanceFlow (BF) & medium & acceptable & yes & no & yes \\
GraphFlow (GF) & easy & fast & yes(opt) & yes & yes
\end{tabular}
\caption{\textbf{Models summary.} The qualitative attributes are relative, e.g., the GraphFlow presents the lowest running time while LocalSearch presents the highest.}
\end{table}}

\onslide<2->{
\begin{figure}
\center
\captionsetup{type=table}
\footnotesize
\begin{tabular}{|l|c|c|c|c|c|}
\hline
& Pixels & LocalSearch & FlipFlow & BalanceFlow & GraphFlow \\
\hline
Triangle & 8315 & 4.8s/it & 0.4s/it & 0.38s/it & 0.14s/it\\
Square & 12769 & 2s/it & 0.51s/it & 0.47s/it & 0.12s/it\\
Ellipse  & 10038 & 3.1s/it & 0.64s/it & 0.57s/it & 0.1s/it \\
Flower & 26321 & 12.3s/it & 1.23s/it & 0.94s/it & 0.14s/it\\
Bean  & 25130 & 6.4s/it & 1.2s/it & 1.17s/it & 0.16s/it\\
\hline
\end{tabular}
\caption{\textbf{Exp-General summary.} Running time and input size of Exp-General experiment for the free elastica.}
\end{figure}}

\end{frame}

\begin{frame}
{Conclusion}
{Summary of models}

\begin{itemize}
\item{Four discrete local models of elastica minimization based on multigrid convergent estimators.}
\pause
\item{Extra terms, such as data fidelity, can be attached to FlipFlow/BalanceFlow and GraphFlow models.}
\pause
\item{Contour completion can be recovered in some cases.}
\pause
\end{itemize}

\textbf{Pros}
\begin{itemize}
\item{No model of curve.}
\pause
\item{Parellizable.}
\pause
\item{Neighborhood flexibility.}
\pause
\end{itemize}

\textbf{Cons}
\begin{itemize}
\item{Susceptible to bad local minimum.}
\pause
\item{Contour completion is difficult to recover and usually more expensive.}
\end{itemize}

\end{frame}

\begin{frame}
{Conclusion}
{Perspectives}

\begin{itemize}
\item{GraphFlow and perimeter: enrich the cost function of GraphFlow with the weights defined in~\cite{boykov03geodesics}. }
\pause
\item{Different neighborhoods: random, linear extension.}
\pause
\item{Dynamic radius: use MDCA (parameter free) to adapt the estimation disk radius to use.}
\pause
\item{Multiresolution: Improve running time; or improve estimator precision.}
\pause
\item{Global optimization and multigrid convergent estimators: Do there exist a practicable model for elastica?}
\end{itemize}

\end{frame}

\begin{frame}
\huge
\center
Thank you!
\end{frame}


\begin{frame}[allowframebreaks]
    \frametitle{References}	
    \printbibliography
\end{frame}


\end{document}