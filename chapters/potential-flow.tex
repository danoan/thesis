\chapter{PotentialFlow}
\label{chapter:potential-flow}


In this chapter we introduce the concept of unbalance coefficient which motivates the definition of a new family of energies to regularize the squared curvatute: the $m$-PotentialFlow. We are going to show that the PotentialFlow is closely related to the FlipFlow energy, but with a easier interpretation and easier to implement algorithm.


\section{Potential Elastica}
We start by defining the concept of unbalance coefficient.

\begin{definition}{Unbalance coefficent}
Given digital shape $S \in \Omega$, positive number $r$ and point $p \in \Omega$, the \emph{unbalance coefficient} of $S$ at $p$ is defined as

\begin{align*}
	u(S,p) = ( \frac{\pi r^2}{2} - |B_r(p) \cap S| )^2.
\end{align*}

\end{definition}

The unbalance coefficient definition is very similar to the II squared curvature estimator, except by the scaling factor $\frac{9}{R^6}$ and its domain. The unbalance coefficient is defined everywhere and not only on the shape boundary. Let $F=B_r(p) \cap S$ and $G=B_r(p) \setminus F$. The unbalance coefficient is symmetric with respect $F$ and $G$. 

\begin{align*}
	u(S,p) &= ( \frac{\pi r^2}{2} - |F| )^2 = ( -\frac{\pi r^2}{2} + |G| )^2 = ( \frac{\pi r^2}{2} - |G| )^2.
\end{align*}

The unbalance coefficient is used to estimate the effect of moving the estimation ball center towards the interior or the exterior of the shape. For example, consider the figure \ref{fig:unbalance-plot} in which we plot the unbalance coefficients of points along the diagonal of a square shape. 

\begin{figure}[h!]
\center
\begin{minipage}{0.25\textwidth}
\subfloat{
\includegraphics[scale=1.0]{figures/chapter7/distant-disks-1.png}}\\%
\subfloat{
\includegraphics[scale=1.0]{figures/chapter7/distant-disks-2.png}}\\%
\subfloat{
\includegraphics[scale=1.0]{figures/chapter7/distant-disks-3.png}}
\end{minipage}%
\begin{minipage}{0.75\textwidth}
\includegraphics[scale=0.75]{figures/chapter7/unbalance-coefficient-p2-with-sum.png}
\end{minipage}
\caption{The unbalance coefficient of equidistant points from the shape contour. The red curve indicates the difference between the blue and the orange curve. The intersection point with the $x$-axes or the inflexion point in the red curve indicates the best center point for the estimation ball at the corner. }
\label{fig:unbalance-plot}
\end{figure}


The unbalance coefficient grows if the estimation ball is moved towards the exterior and decreases if the estimation ball is moved towards the interior of the shape. That is an indication that we should remove point $p$ from $S$ to decrease the squared curvature of the shape at point $p \in \partial S$. The same point $p$ may be touched by several balls, each of them contributing with some value for the ultimate decision of keep or remove point $p$ of $S$. 
%The $m$-potential is defined as the sum of these contributions


%\begin{definition}{m-potential}
%Given digital shape $S$, integer numbers $r>0, m \neq 0$, the \emph{m-potential} at point $p$ is defined as
%
%\begin{align*}
%	U_{m}(S,p) &= \sum_{q \in Q_m(p)}{ u(S,q)}
%\end{align*}
%
%where its zone of influence $Q_m(p)$ is defined as
%
%\begin{align*}
%	Q_m(p) &= \left\{\; q \in L_m(S) \; | \; p \in B_r(q) \; \right\} \\
%\end{align*}
%
%\end{definition}
%
%\begin{figure}[h!]
%\center
%\includegraphics[scale=0.5]{figures/chapter7/k-potential.png}
%\caption{The $q$ points of the $m$-ring forms the zone of influence of the boundary point $p$.}
%\label{fig:unbalance-set}
%\end{figure}
%
%The zone of influence of $p$ is the set of all points $q$ in which $p$ is contained in the estimation ball centered at $q$. In the next section we define a flow that will have the following behaviour
%
%\begin{align*}
%	U_m(S^{(k)},x_j) - U_{-m}(S^{(k)},x_j) > 0 &\rightarrow x_j \notin S^{(k+1)} \\
%	U_m(S^{(k)},x_j) - U_{-m}(S^{(k)},x_j) < 0 &\rightarrow x_j \in S^{(k+1)}
%\end{align*}



We are going to consider the inner and outer pixel boundary simultaneously as the optimization variables. We recall that $S^{(k)}$ corresponds to the digital shape $S$ after the $k$th iteration of the flow and that $d_{S^{(k)}}$ is its signed distance transform. Assuming a $m$-ring configuration, the sets we are going to need are summarized below:

\begin{align*}
	O^{(k)} &:=\left\{ p \in \Omega \; | \; -m <= d_{S^{(k)}}(p) \leq m \right\}\\
	X^{(k)} &:= X(O^{(k)})  \\
	F^{(k)} &:= S^{(k)} \setminus O^{(k)} \\
	F_r^{(k)}(p) &:= F^{(k)} \cap B_r(p)\\
	O_r^{(k)}(p) &:= O^{(k)} \cap B_r(p) \\
	X_r^{(k)}(p) &:= X\big( O_r^{(k)}(p) \big).
\end{align*}

We assume $k=0$ if not defined, i.e., $S=S^{(0)}$. We denote $p_i, p_o \in \mathcal{R}_m(S)$ as the inner and outer ball centers in the $m$-ring of $S$. To simplify notation, let $X_o=X_r(p_o)$ and $X_i=X_r(p_i)$. We define the term $T_{m-pot}(S)$ as

\begin{align}
	T_{m-pot}(S) &= \sum_{p_i,p_o \in \mathcal{R}_m(S)}{( \frac{\pi R^2}{2} - (\; |F_o| + \sum_{ x_j \in X_o}{1-x_j} \; ) )^2 -(\frac{\pi R^2}{2} - (\; |F_i| + \sum_{x_j \in X_i}{x_j}\;))^2}.
	\label{eq:potential-term}
\end{align}

Locally, labeling $x_j=0$ implies to change the initial unbalance of the outer ball, while keeping the initial unbalance of the inner ball unchanged. 

The $m$-potential energy family is defined as

\begin{align}
  E_{m-pot}(X^{(k)},S^{(k)}) =& \sum_{x_j \in X^{(k)}}{\alpha s(x_j)} + \beta T_{m-pot}(S^{(k)}) + \sum_{x_j \in X^{(k)}}{\gamma g(S^{(k)},x_j)}.
  \label{eq:single-step-energy-family}
\end{align}

We follow the same notation of chapter \ref{chapter:flip-flow} to denote the data term $g(S,x)$ and the length penalization term $s(x)$ defined below. 

\begin{align}
  s(x_{w(p)})=\sum_{q \in \mathcal{N}_4(p)}{ t(q) }, \quad \text{where } t(q) = \left\{\begin{array}{ll}
  (x_{w(p)}-x_{w(q)})^2, & \text{if } q \in O^{(k)}\\
  (x_{w(p)}-1), & \text{if } q \in F^{(k)}\\
  (x_{w(p)}-0), & \text{otherwise }
  \end{array}\right.
  \label{eq:single-step-length-penalization}
\end{align}

\begin{align}
  g(x_{w(p)}) = -x_{w(p)}\log{H_f(p)} - (1-x_{w(p)})\log{H_b(p)},
  \label{eq:single-step-data-fidelity}
\end{align}	

where $H_f ,H_b $ are the mixed Gaussian distributions of color intensities derived from the foreground and background seeds. The PotentialFlow algorithm is summarized in \ref{alg:potential-flow}.

\begin{algorithm}
 \SetKwData{It}{k}
 \SetKwData{MIt}{maxIt}
 \SetKwData{Delta}{delta}
 \SetKwInOut{Input}{input}\SetKwInOut{Output}{output}
 \SetKwComment{comment}{//}{}
 
 \Input{A digital set $S$; The ring number $m$; Length($\alpha$), curvature($\beta$) and data($\gamma$) coefficients; the maximum number of iterations \MIt;}
 \BlankLine
 $S^{(0)} \longleftarrow S$\;
 $k \longleftarrow 1$\;
 \While{ \It $<$ \MIt  }{ 	
	 	$x^{(k-1)} \longleftarrow \argmin_{X^{(k-1)}} E_{m-pot}(X^{(k-1)},S^{(k-1)})$\; 	
 		$S^{(k)} \longleftarrow F^{(k-1)} + x^{(k-1)}$\;
 	
	\It $\longleftarrow$ \It $+1$\;
	
 }
 \caption{PotentialFlow algorithm.}
 \label{alg:potential-flow}  
\end{algorithm}


\section{Relation with FlipFlow}
	The PotentialFlow returns similar solutions to the FlipFlow algorithm. Indeed, they are closely related. We recall the curvature regularization term of the FlipFlow energy \eqref{eq:energy-family} for some digital shape $S$. 

\begin{align}
T_{m-flip}(S) &= \sum_{ p_i,p_o \in R_m(S)}{ ( \frac{\pi R^2}{2} - F_r(p_o) - \sum_{x_j \in X_r(p_o)}{x_j})^2 + (\frac{\pi R^2}{2} - F_r(p_i) - \sum_{x_j \in X_r(p_i)}{x_j})^2 }
\label{eq:curvature-term}
\end{align}

We replace $x_j$ to $(1-x_j)$, which corresponds to take the complement of the solution in the FlipFlow model. To simplify notation, we are going to ommit the radius $r$ and replace points $p_i,p_o$ by an underscored $i,o$, i.e., $F_i := F_r(p_i)$. We rewrite \eqref{eq:curvature-term} as

\begin{align}
T_{m-flip}(S) &= \sum_{ p_i,p_o \in R_m(S)}{ ( \frac{\pi R^2}{2} - F_o - \sum_{x_j \in X_o}{1-x_j})^2 + (\frac{\pi R^2}{2} - F_i - \sum_{x_j \in X_i}{1-x_j})^2 }
\label{eq:curvature-term-simplification}
\end{align}

Next, let $A_i = \pi R^2/2 - |F_i|$. We rewrite  the second term of \eqref{eq:curvature-term-simplification} as

\begin{align*}
	(A_i - \sum_{x_j \in X_i}{ (1-x_j) })^2 &= (A_i - |X_i| + \sum_{x_j \in X_i}{ x_j })^2 \\
	&= (A_i - |X_i|)^2 + 2(A_i - |X_i|)\sum_{x_j \in X_i}{x_j} + ( \sum_{x_j \in X_i}{x_j} )^2\\	
	&= A_i^2 -2A_i|X_i| + |X_i|^2 + 2(A_i - |X_i|)\sum_{x_j \in X_i}{x_j} + ( \sum_{x_j \in X_i}{x_j} )^2\\
	&= A_i^2 + 2A_i\sum_{x_j \in X_i}{x_j} + ( \sum_{x_j \in X_i}{x_j} )^2 - 2A_i|X_i| + |X_i|^2 -2|X_i|\sum_{x_j \in X_i}{x_j} \\
	&= 2A_i^2 - (A_i - \sum_{x_j \in X_i}{x_j})^2 + 2( \sum_{x_j \in X_i}{x_j} ) ^2 - 2A_i|X_i| + |X_i|^2 - 2|X_i|\sum_{x_j \in X_i}{x_j}
\end{align*}

	We group the constants into the constant term $c=2A_i^2 - 2A_i|X_i|$	 to obtain
\begin{align}
		(A_i - \sum_{x_j \in X_i}{ (1-x_j) })^2 &= - (A_i - \sum_{}{x_j})^2 + (|X_i| - \sum_{x_j \in X_i}{x_j})^2 + (\sum_{x_j \in X_i}{x_j})^2 + c \nonumber \\
	&= - (A_i - \sum_{x_j \in X_i}{x_j})^2 + P_1(X_i) + c \nonumber \\
	&= - (\frac{\pi R^2}{2} - (|F_i| + \sum_{x_j \in X_i}{x_j}) )^2 + P_1(X_i) + c \nonumber 	
	\label{eq:second-term}
\end{align}

where $P_1(X_i) = (\sum_{}{ 1-x_j})^2 + (\sum_{}{x_j})^2$. Finnaly, we  replace \eqref{eq:second-term} in \eqref{eq:curvature-term-simplification} to obtain

\begin{align}
T_{m-flip}(S) &= ( \frac{\pi R^2}{2} - (\; |F_o| + \sum_{ x_j \in X_o}{1-x_j} \; ) )^2 -(\frac{\pi R^2}{2} - (\; |F_i| + \sum_{x_j \in X_i}{x_j}\;))^2  + P_1(X_i) + c \nonumber \\
&= T_{m-pot}(S) + P_1(X_i) + c.
\end{align}

The two energies differ in the penalty term $P_1(X_i)$. Such term favors solutions in which half of the variables are labeled one. The FlipFlow and PotentialFlow behave similarly. For lower values of $m$ both produ However, it's more likely that the PotentialFlow converges to an intermediate shape instead of evolving to a single point (see figure \ref{fig:potential-flow-flip-flow-comparison}). We believe that the extended optimization region is the reason for that.

\begin{figure}
\begin{tabular}{cccc}
$7$-PotentialFlow & \includegraphics[scale=0.25]{figures/chapter7/potential-flow/triangle/summary.pdf} & \includegraphics[scale=0.25]{figures/chapter7/potential-flow/flower/summary.pdf} & \includegraphics[scale=0.25]{figures/chapter7/potential-flow/bean/summary.pdf} \\
$7$-FlipFlow & \includegraphics[scale=0.25]{figures/chapter7/flip-flow/triangle/summary.pdf} & \includegraphics[scale=0.25]{figures/chapter7/flip-flow/flower/summary.pdf} & \includegraphics[scale=0.25]{figures/chapter7/flip-flow/bean/summary.pdf} \\
\end{tabular}
\caption{Evolutions of three shapes for the $7$-PotentialFlow and $7$-FlipFlow using a estimation ball radius of $9$. The models present similiar behaviour. Shapes are displayed at every $10$ iterations.}
\label{fig:potential-flow-flip-flow-comparison}
\end{figure}

\section{Graph-based formulation}

Let $A_o = \pi R^2 / 2 - |F_o|$ and $A_i = \pi R^2/2 - |F_i|$. Putting $x_j$ in evidence we can rewrite \eqref{eq:potential-term} as

\begin{align*}
	T_{m-pot}(S) &= (A_o - |X_o|)^2 - A_i^2 + \sum_{x_j \in X}{x_j\Delta_j T_{m-pot}(S)}
\end{align*} 

Grouping the constants in $c=(A_o - |X_o|)^2 - A_i^2$ we obtain
\begin{align}
	T_{m-pot}(S) &= c +\sum_{x_j \in X}{ 2x_j( A_o - |X_o| + \frac{1}{2}\sum_{x_l \in X_o}{x_l} + A_i + \frac{1}{2}\sum_{x_l \in X_i}{x_l})} \nonumber \\
	&= c +\sum_{x_j \in X}{2x_j( A_o - \sum_{x_l \in X_o}{1-\frac{x_l}{2}} + A_i - \sum_{x_l \in X_i}{\frac{x_l}{2}})}.
	\label{eq:xj-evidence}
\end{align}

We remark the similarity of the expression between parentheses in \eqref{eq:xj-evidence} and the sum $u(S,p_o)^{1/2} + u(S,p_i)^{1/2}$, i.e., the sum of unbalance coefficients square roots. The variable $x_j$ is set to zero if the expression in parentheses is less than zero. That remark motivates us to define a graph-based model, described in the next chapter, in which the edge's weight are given by the sum of unbalance coefficients.


\section{Conclusion}
The PotentialFlow model is closely related to the FlipFlow but it has a simple implementation and interpretation. The unbalance coefficent is the link that connects both models and the motivation for the graph-based model proposed in the next chapter.




