\chapter{Constraint Model}
\label{chapter:appendix-constraint-model}

In this chapter we describe a plenty of models that target to solve the third order energy. The basic strategies are convexification and linearization.

\textbf{Obs:} The interpixel model is not the best choice for doing the global optimization. Think on the square and an estimation ball of radius $3$. In the interpixel mode, such ball would have radius $6$ and it would touch $8$ pixels in the trust foreground region. Summing up the pixels in the optimization region we get $14$ pixels. For the radius $3$ ball, equilibrium is reached with $14.5$. We can argue that $14$ is closer from the average than the $18$ we have in the pixel mode, but we can get $14.5$ exactly in the average pixel mode. Discussion should be done over the average pixel mode instead.


\textbf{Obs:} While doing binary optimization, we can write $x^2$ as $x$, it doesn't change the final result. But if we wish to the some sort of relaxation and convex optimization, we need to properly count the squared term, as it has an influence on the pairwise matrix being semidefinite positive or not.


\textbf{Obs:} Rewrite model using matrix notation. Let $x_i \in \mathbb{R}^n$ the active variables when ball is positioned at point $p_i$. Then,

\begin{align*}
	f(x) =& \sum_{p_i \in A}{ \big(C_i - x_i^Tx_i\big)^2} \\
	=& \sum_{p_i \in A}{ C_i^2 - 2C_ix_i^Tx_i + x_i^Tx_ix_i^Tx_i } \\
	=& q^tx + x^tQx + C 
\end{align*}
, where $\displaystyle C_i = \frac{\pi}{r^2} - F_i, \,q = \sum_{p_i \in A}{-2C_ix_i},\, Q = \sum_{p_i \in A}{x_ix_i^T}$.

Such problem is convex, therefore, we can apply the round-down, round-up procedures in order to produce optimal binary solutions, given the optimal relaxed convex solution.

\textbf{Outline:} 

\begin{enumerate}
	\item{Prove third order energy is non-convex.}
	\item{Define two convexification strategies.}
	\item{Be precise on how Lagrangian multipliers can help us.}
\end{enumerate}

\textbf{General Model:} 	Consider the standard cell grid model (Khalimsky space) with its $N$ cells and $L$ linels. Assume a clockwise orientation for cells; up orientation for vertical linels; and right orientation for horizontal linels. The orientation is arbitrary. 
	
	Pixel variables are denoted by $v \in \{0,1\}^N$. Linel variables by $w \in \{0,1\}^{2L}$. Note that each linel generates two binary variables, one for each direction. We called an oriented linel an edge.
			
	Further, we define incidence relations between cells and linels and between edges and linels.
	
	\[ \forall 0 \leq j < N, \quad
			\alpha_{j,i} = \left\{ 		\begin{array}{ll}
											+1&, \text{ if cell $j$ is positively incident to linel i }\\
											-1&,\text{ if cell $j$ is negatively incident to linel i }\\
											0&,\text{otherwise}.						
										\end{array}\right.
	\]
	
	\[\forall 0 \leq j < 2L, \quad
			\beta_{j,i} = \left\{ 		\begin{array}{ll}
											+1&, \text{ if edge $j$ is positively incident to linel i }\\
											-1&,\text{ if edge $j$ is negatively incident to linel i }\\
											0&,\text{otherwise}.						
										\end{array}\right.
	\]	

	The set of consistent constraints is simply expressed as:
	\begin{align*}
		\sum_{}{\alpha_{j,i}v_i} + \sum_{}{\beta_{j,i}v_i} = 0
	\end{align*}
	
	In words, the sum of cells incidences and edges incidences to some linel $v_i$ should be equal to zero. 
	
	We can wrap up the variables in one big variable $y \in \{0,1\}^{N+2L}$ and the incidence coefficients in a single matrix $A$. The problem formulation is finally written as


	\begin{equation}
	\begin{aligned}
		y^\star &= \argmin_{y} \sum_{w_i \in \mathcal{L}}{ a^2 \cdot \left( C_0 + C_1 \cdot \sum_{v_j \in \mathcal{C}}{v_j} + 2 \cdot \sum_{ \substack{v_j,v_k \in \mathcal{C} \\ j<k} }{v_jv_k}  \right) } \cdot w_i,\\[1em]
	&\quad \text{ subject to } Ay=0.
	\end{aligned},
	\label{eq:global-formulation}
	\end{equation}
	
	with 
	\begin{align*}
		a &= 3/r^3 \\ 
		b &= \pi r^2/2 \\ 
		C_0 &= b^2 - 2b \cdot |F_r(w_i)| + |F_r(w_i)|^2 \\
		C_1 &= 2\left( |F_r(w_i)| - b + 1/2 \right).
	\end{align*}


\section{General Optimization: BFGS}
Since the problem is non-convex, this approach is expected to find a local minimum at most. However, if we have some stragegy that give us warm-starters, we can expect to find solutions not too far from optimal.

\section{Cubics Linearization and Convexification}
The strategy here is to quadrify the energy. A straightforward linearization of cubic terms is not sufficient because this leads to a quadratic energy with non-semidefinite positive matrix, which eliminates methods from convex optimization.

We need to apply some sort of energy convexification as well, in the hope that the global optimium of the resulting convexification energy is not far away from the global optimal.
	
	\begin{equation}
	\begin{aligned}
		\min &\frac{1}{2} y^TPy + q^Ty\\
		st \quad &Gy + s = h\\
		&Ay = b \\
		& s \geq 0
		\label{eq:quadratic-formulation}
	\end{aligned}
	\end{equation}
	
	In order to do that, we need to choose a second order term to linearize. We can linearize $v_jv_k$ or $v_jw_i$. However, neither will lead to a semi-definite positive matrix $P$, in which case we could solve using techniques from convex optimization.

	\textbf{Open question 1}: How far is the matrix $P$ from being semi-definite positive for the choice $v_jw_i$ (all coefficients positive)? What about the choice $v_jv_k$? \\
		
	\textbf{Open question 2}: Is there a way to compute an approximation from my energy that leads to a semi-definite positive matrix $P$?
	
	\textbf{Strategy 1:} Variable substituion plus constraint linearization.
	
	\textbf{Strategy 2:} Quadratic reduction by penalization (following Boros) + convexification.
	
	\textbf{Strategy 3:} Strategies 1 and 2 should be evaluated under estimation ball on linels model and estimation ball average over incident pixels models.
	
	\textbf{Strategy 4:} Use max-energy to define a global optimization energy (it is simply necessary to condition the estimation ball evaluations to boundary linels).
	
	
\section{Lagrangian Multipliers}

	We can try variations of the model by including constraints as penalty terms in the objective function.
	
	
\section{Augmented Lagrangian Multipliers}	

