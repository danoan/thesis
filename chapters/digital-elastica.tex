\chapter{A combinatorial model for digital Elastica shape optimization}
\label{chapter:digital-elastica}

In this chapter we review the Elastica energy and some of its properties. Next, we introduce the digital version of the Elastica using multigrind convergent estimators of length and curvature. We present a combinatorial optimization model capable to evolve a shape to another of lower digital Elastica value. In several occasions, the final shape is indeed the optimal one, which confirms the pertinence of using multigrid convergent estimators to optimize geometric-related energies in digital sets. Finally, we present several attempts to derive a global model to minimize a simplification of the digital Elastica and we discuss why they fail.

\section{Continuous and digital Elastica}
\label{ch6:sec:continuous-digital-elastica}
	\sketch{To be further developed...}

	The Elastica energy of parameters $\Theta=(\alpha, \beta) \geq 0$ for some Euclidean shape $S \subset \mathbb{R}^2$ is defined as
	
	\begin{align*}
	E_{\Theta}(S) &= \int_{\partial S}{ \alpha + \beta \kappa(s)^2 ds}.
	\end{align*}
	
	
On the other hand, the digital Elastica $\hat{E}_{\Theta}$ of some digitization $D_h(S)$ of $S$ is defined as
	\begin{align}
	\hat{E}_{\Theta}( D_h(S) ) = \sum_{\dot{\vec{e}} \in \partial D_h(S)}{ \hat{s}( \dot{\vec{e}})\left(\; \alpha + \beta \hat{\kappa}_{r}^2(D_h(S),\dot{\vec{e}},h) \; \right)},
	\label{eq:digital-elastica}
	\end{align}
	
where $\dot{\vec{e}}$ denotes the center of the edge $\vec{e}$. In the
expression above, we will substitute an arbitrary subset $\Ds$ of
$\mathbb{Z}^2$ to $D_h(S)$ since the continuous shape $S$ is unknown.
In the following we omit the grid step $h$ to simplify expressions
(or, putting it differently, we assume that the shape of interest is
rescaled by $1/h$ and we set $h=1$). 

In the next section, we describe a combinatorial scheme that permit us to find the minimum digital shape with respect the digital Elastica energy for some neighborhood of shapes of $\Ds$. 

\section{Local combinatorial scheme}
\label{ch6:sec:local-combinatorial-scheme}

Given a digital shape $\Ds^{(0)}$ we describe a process that generates a
sequence $\Ds^{(i)}$ of shapes with non-increasing Elastica energy. The
idea is to define a neighborhood of shapes $\mathcal{N}^{(i)}$ to the
shape $\Ds^{(i)}$ and choose the element of $\mathcal{N}^{(i)}$ with
lowest energy.  The process is suited for the integral invariant
estimator but also for other curvature estimators, for example, MDCA
\cite{roussillon11mdca}. As a matter of fact, our experiments have
shown that either estimators induce similar results.

Let $\Ds$ be a $2$-dimensional digital shape embedded in a domain $\Omega \subset \mathbb{Z}^2$. We adopt the cellular-grid model to represent $\Ds$, i.e., pixels and its lower dimensional counterparts, linels and pointels, are part of $\Ds$ (see~\cref{fig:cellular-grid-model}). In particular, we denote by $\partial \Ds$ the topological boundary of $\Ds$, i.e., the connected sequence of linels such that for each linel we have one of its incident pixels in $\Ds$ and the other not in $\Ds$.

\begin{figure}[]
	\center
	\subfloat[\label{}]{%
	\includegraphics[scale=1.8]{figures/chapter5/cellular-grid/flower.png}
	}\hspace{40pt}%
	\subfloat[\label{}]{%	
	\includegraphics[scale=0.27]{figures/chapter5/cellular-grid/cellular-grid-illustration.pdf}
	}\hspace{40pt}%
	\subfloat[\label{}]{%
	\includegraphics[scale=0.035]{figures/chapter5/gcurves/distance-transform.pdf}
	}
	\caption{The flower shape in figure (a) and the cellular-grid model representation in (b) of the rectangle-bounded region. In figure (b), pixels are colored in gray, linels in green and pointels in blue. In figure (c), the blue pixels denotes a $3$-ring set.}
	\label{fig:cellular-grid-model}
\end{figure}

Let $d_{\Ds}:\Omega \rightarrow \mathbb{R}$ be the signed Euclidean distance transformation with respect to shape $\Ds$. The value $d_{\Ds}(p)$ gives the Euclidean distance between $p \notin \Ds$ and the closest pixel in $\Ds$. For points $p \in \Ds$, $d_{\Ds}(p)$ gives the distance between $p$ and the closest pixel not in $\Ds$.

\begin{definition}{m-Ring Set}
Given a digital shape $\Ds \in\Omega$, its signed distance transformation $d_{\Ds}$ and natural number $m \neq 0$, the {\em $m$-ring set of $\Ds$} is defined as
\begin{align*}
	R_m(\Ds) &:= L_m \cup L_{-m},
\end{align*}
where
\begin{align*}
	L_m(\Ds) &:= \left\{ \quad \begin{array}{cc}
		\left\{ p \in \Omega \; | \; m-1 < d_{\Ds}(p) \leq m \right\} & , \quad m>0\\
		\left\{ p \in \Omega \; | \; m+1 > d_{\Ds}(p) \geq m \right\} & , \quad m<0
		\end{array} \right.
\end{align*}
\end{definition}


Consider the following set of neighbor candidates to $\Ds$:
\begin{align*}
\mathcal{U}(\Ds) = \{ Q \; | \; Q \subset R_1(\Ds) \cup \Ds \; \text{and} \; \text{$Q$ is connected} \}.
\end{align*}


Such set can be extremely large and its complete exhaustion is prohibitively expensive.  Instead, we are going to use a subset of it.


\begin{definition}{$n$-neighborhood}

	Given a digital shape $\Ds \in \Omega$, its $n$-neigh\-bor\-hood $\mathcal{N}_n(\Ds)$ is defined as the set of digital shapes that can be built from $\Ds$ by adding or removing a sequence of $k \in [0,n]$ connected pixels in $R_1(\Ds)$.

\end{definition}



At first glance, we may be tempted to set the local-search neighborhood at the $k$-th iteration as the union of all $n$-neighborhood for $1<n<|\partial \Ds^{(k)}|$. However, that is often unecessary and timing consuming, as the greatest reduction in digital Elastica for a member of $\mathcal{N}_n$ is likely very close to the greatest reduction for a member of $\mathcal{N}_{n-1}$. Moreover, we can improve running time by implementing a multiscaling approach, i.e., we look for reductions in digital Elastica for larger values of $n$ first, and in case of a negative answer we refine our search by choosing a smaller $n$.

~\cref{alg:local-search} describes the local combinatorial process and is suitable for any type of digital estimator. To estimate length we use MDSS and to estimate curvature we execute~\cref{alg:local-search} with the MDCA and II-$r$ estimators ($r$ denoting the radius of the estimation ball) to solve the free and constrained Elastica problems. 
	

\begin{figure}[]
\center
\subfloat{
\includegraphics[scale=0.4]{figures/chapter5/gcurves/gc/main-inner.pdf}
}\hspace{1em}%
\subfloat{
\includegraphics[scale=0.4]{figures/chapter5/gcurves/gc/main-outer.pdf}
}\hspace{1em}\\[1em]
\subfloat{
\includegraphics[scale=0.4]{figures/chapter5/gcurves/gc/inner-main.pdf}
}\hspace{1em}%
\subfloat{
\includegraphics[scale=0.4]{figures/chapter5/gcurves/gc/outer-main.pdf}
}\hspace{1em}%
\caption{Members of $\mathcal{N}_{11}$ for the square shape.}
\end{figure}


\begin{algorithm}[H]
 \SetKwData{It}{k}
 \SetKwData{MIt}{maxIt}
 \SetKwData{Delta}{delta}
 \SetKwData{Best}{best} 
 \SetKwInOut{Input}{input}\SetKwInOut{Output}{output}
 \SetKwComment{comment}{//}{}
 
 \Input{A digital set $\Ds$; weight coefficients $\Theta=(\alpha , \beta)$; the number of curve segments $nc$; the maximum number of iterations \MIt}
 \BlankLine
 $t \longleftarrow 1$ \tcp*[l]{level of multiscaling}
 $k \longleftarrow 0$	\tcp*[l]{current iteration}
 $\Ds^{(0)} \longleftarrow \Ds$\;
 \While{ \It $<$ \MIt \bf{and} $t < \log_2{|\partial \Ds^{(k)}|}$ }
 { 	
	$M^{(k,t)}  \longleftarrow |\partial \Ds^{(k)}|/2^t$  	\tcp*[l]{Maximum $n$-neighborhood value.}
	
	$J \longleftarrow \{ j \cdot \frac{M^{(k,t)}}{nc} \; | \; 1 \leq j \leq nc \} $\;
	
	$\displaystyle \mathcal{N}^{(k,t)} \longleftarrow \displaystyle \Cup_{J}{\mathcal{N}_{j}(\Ds^{(k)})}$\;
	\BlankLine
	
	\comment{Find neighbor shape with lowest energy.} 
	$Q^{\star} \longleftarrow D^{(k-1)}$\;
  	\For{$ Q \in \mathcal{N}^{(k,t)} $}
	{
		\If{ $\hat{E}_{\Theta}(Q)$ $<$ $\hat{E}_{\Theta}(Q^\star)$ }
		{
			$Q^\star \longleftarrow Q$\; 
		}	
	}
	\BlankLine
	

	\Delta $\longleftarrow$ $\hat{E}_{\Theta}(\Ds^{(k-1)}) - \hat{E}_{\Theta}(\Ds^{(k)})$\;	
	
	\If{ \Delta $\leq 0$ } 
	{
		\comment{Better solution not found. Refine the scale.}
		$t \longleftarrow t+1$ 
	}		
	\Else 
	{ 
		\comment{Better solution found. Set $D^{(k)}$ and reset multiscaling.}
		$t \longleftarrow 1$\;	
		$\Ds^{(k)} \longleftarrow Q^\star$\;
		\It $\longleftarrow$ \It $+1$\;		
	}
	
 }
 \caption{LocalSearch algorithm for Elastica minimization.}
  \label{alg:local-search} 
\end{algorithm}

\subsection{Free digital Elastica}
\label{ch6:subsec:free-digital-elastica}
In the free digital Elastica energy we optimize~\cref{eq:digital-elastica} without any constraint. We observe that for $\alpha=0, \beta >0$ the Elastica becomes the integration of the squared curvature  along the shape contour which has the ball of infinite radius as its minimizer. For $\alpha > 0, \beta=0$, minimize Elastica becomes minimize perimeter (curvature flow). It is easy to see that for $\alpha, \beta > 0$, the optimal shape for the Elastica is a disk of radius $r$. We can easily find the value of $r$.

\begin{align*}
	\frac{d}{dr}\big( \int_{\partial B(r)}{ (\alpha + \beta \kappa ^2) ds} \big ) &= 0\\
	\frac{d}{dr}\big( \alpha 2\pi r + \beta \pi/r \big) &= 0\\
	r &= \alpha^{-1/2}.
\end{align*}  

Therefore, the optimal shape for the free digital Elastica is a digital disk of finite radius $\alpha^{-1/2}$.


In~\cref{fig:local-comb-estimators-plots-lp001} we present the digital Elastica evolution for parameters $\alpha=0.01, \beta=1$ and three different curvature estimators in three different scales. The shapes evolution using the II-$5$ estimator are shown in~\cref{fig:local-comb-ii5-results}. We observe that both II-$5$ and II-$10$ evolve the shapes to disks of radius close to the optimum value of $10$. The II-$3$ estimator stops prematurely at a local optimum due its limited sensibility compared to II-$5$ or II-$10$, while MDCA encounters some difficults to evolve in a high resolution setting and it also stops at some local optimum. In fact, the MDCA estimator, although with higher convergence speed, is more sensitive to noise than II, as illustrated in~\cref{fig:mdca-sensitivity}. Nonetheless, the results can be improved by using a larger neighborhood, as illustrates~\cref{fig:mdca-larger-neighborhood}.



We have executed the same experiments for different parameters $\alpha$ to confirm the effectiveness of our approach. We observe that the plots for $\alpha=0.001$ in~\cref{fig:local-comb-estimators-plots-lp0001} follows a pattern similar to those in~\cref{fig:local-comb-estimators-plots-lp001} for $\alpha=0.01$. In particular, the remarks for the II-$3$ and MDCA estimator are the same. Further, we point out that II-$5$ values are slightly farther from the optimum for $\alpha=0.001$. The reason being that the shapes evolve to a ball of higher radius compared to the case $\alpha=0.01$. At some point of the evolution for $\alpha=0.001$, the sensibility of II-$5$ is not sufficient to escape from local optimum. We remark that the adoption of an automatic selection of the estimation ball radius may attenuate this problem.



\begin{figure}[]
\center
\subfloat{
\includegraphics[scale=0.4]{figures/chapter5/flow/plots/bars/length_pen_0.01000/triangle.pdf}
}\hspace{1em}%
\subfloat{
\includegraphics[scale=0.4]{figures/chapter5/flow/plots/bars/length_pen_0.01000/square.pdf}
}\hspace{1em}%
\subfloat{
\includegraphics[scale=0.4]{figures/chapter5/flow/plots/bars/length_pen_0.01000/flower.pdf}
}\hspace{1em}%
\subfloat{
\includegraphics[scale=0.4]{figures/chapter5/flow/plots/bars/length_pen_0.01000/bean.pdf}
}\hspace{1em}%
\subfloat{
\includegraphics[scale=0.4]{figures/chapter5/flow/plots/bars/length_pen_0.01000/ellipse.pdf}
}\hspace{1em}%
\subfloat{
\includegraphics[scale=0.4]{figures/chapter5/flow/plots/summary/lp_0.01/summary-ii5.pdf}
}%
\caption{Minimum value attained for the digial Elastica ($\alpha=0.01, \beta=1$) in comparisson with the global optimium (dashed line) for different curvature estimators and in different scales. The last figure summarizes the digital Elastica evolution value for all shapes using grid step $h=0.25$.}
\label{fig:local-comb-estimators-plots-lp001}
\end{figure}

\begin{figure}[]
\center
\subfloat{
\includegraphics[scale=0.4]{figures/chapter5/flow/plots/bars/length_pen_0.00100/triangle.pdf}
}\hspace{1em}%
\subfloat{
\includegraphics[scale=0.4]{figures/chapter5/flow/plots/bars/length_pen_0.00100/square.pdf}
}\hspace{1em}%
\subfloat{
\includegraphics[scale=0.4]{figures/chapter5/flow/plots/bars/length_pen_0.00100/flower.pdf}
}\hspace{1em}%
\subfloat{
\includegraphics[scale=0.4]{figures/chapter5/flow/plots/bars/length_pen_0.00100/bean.pdf}
}\hspace{1em}%
\subfloat{
\includegraphics[scale=0.4]{figures/chapter5/flow/plots/bars/length_pen_0.00100/ellipse.pdf}
}\hspace{1em}%
\subfloat{
\includegraphics[scale=0.4]{figures/chapter5/flow/plots/summary/lp_0.001/summary-ii5.pdf}
}%
\caption{Minimum value attained for the digial Elastica ($\alpha=0.001, \beta=1$) in comparisson with the global optimium (dashed line) for different curvature estimators and in different scales. The last figure summarizes the digital Elastica evolution value for all shapes using grid step $h=0.25$.}
\label{fig:local-comb-estimators-plots-lp0001}
\end{figure}


\begin{figure}[hp!]
	\center
	\begin{tabular}{ccc}
		$h=1.0$ & $h=0.5$ & $h=0.25$ \\[2em]
	\includegraphics[scale=0.185]{figures/chapter5/flow/triangle/radius_5/ii/elastica/len_pen_0.01000/jonctions_1/curve_segs_4/best/gs_1.00000/summary.pdf} &
	\includegraphics[scale=0.185]{figures/chapter5/flow/triangle/radius_5/ii/elastica/len_pen_0.01000/jonctions_1/curve_segs_4/best/gs_0.50000/summary.pdf} &
	\includegraphics[scale=0.185]{figures/chapter5/flow/triangle/radius_5/ii/elastica/len_pen_0.01000/jonctions_1/curve_segs_4/best/gs_0.25000/summary.pdf}\\[2em]
		
	\includegraphics[scale=0.17]{figures/chapter5/flow/square/radius_5/ii/elastica/len_pen_0.01000/jonctions_1/curve_segs_4/best/gs_1.00000/summary.pdf} &
	
	\includegraphics[scale=0.17]{figures/chapter5/flow/square/radius_5/ii/elastica/len_pen_0.01000/jonctions_1/curve_segs_4/best/gs_0.50000/summary.pdf} &	
	
	\includegraphics[scale=0.17]{figures/chapter5/flow/square/radius_5/ii/elastica/len_pen_0.01000/jonctions_1/curve_segs_4/best/gs_0.25000/summary.pdf}\\[2em]
	
	
	\includegraphics[scale=0.25]{figures/chapter5/flow/flower/radius_5/ii/elastica/len_pen_0.01000/jonctions_1/curve_segs_4/best/gs_1.00000/summary.pdf} &		
	
	\includegraphics[scale=0.25]{figures/chapter5/flow/flower/radius_5/ii/elastica/len_pen_0.01000/jonctions_1/curve_segs_4/best/gs_0.50000/summary.pdf} &

	\includegraphics[scale=0.25]{figures/chapter5/flow/flower/radius_5/ii/elastica/len_pen_0.01000/jonctions_1/curve_segs_4/best/gs_0.25000/summary.pdf}\\[2em]	
	
	\includegraphics[scale=0.25]{figures/chapter5/flow/bean/radius_5/ii/elastica/len_pen_0.01000/jonctions_1/curve_segs_4/best/gs_1.00000/summary.pdf} &	 
	
	\includegraphics[scale=0.25]{figures/chapter5/flow/bean/radius_5/ii/elastica/len_pen_0.01000/jonctions_1/curve_segs_4/best/gs_0.50000/summary.pdf} &	
	
	\includegraphics[scale=0.25]{figures/chapter5/flow/bean/radius_5/ii/elastica/len_pen_0.01000/jonctions_1/curve_segs_4/best/gs_0.25000/summary.pdf}\\[2em]			

	
	\includegraphics[scale=0.25]{figures/chapter5/flow/ellipse/radius_5/ii/elastica/len_pen_0.01000/jonctions_1/curve_segs_4/best/gs_1.00000/summary.pdf} &

	\includegraphics[scale=0.25]{figures/chapter5/flow/ellipse/radius_5/ii/elastica/len_pen_0.01000/jonctions_1/curve_segs_4/best/gs_0.25000/summary.pdf} &

	\includegraphics[scale=0.25]{figures/chapter5/flow/ellipse/radius_5/ii/elastica/len_pen_0.01000/jonctions_1/curve_segs_4/best/gs_0.25000/summary.pdf}				
\end{tabular}
		\caption{LocalSearch algorithm evolutions for several shapes with $\alpha=0.01,\beta=1$. The initial contour is colored in red; the final contour is colored in blue; and the optimal contour is colored in green.}	
		\label{fig:local-comb-ii5-results}
\end{figure}

\begin{figure}
\center
\begin{tabular}{cccc}
& $h=1.0$ & $h=0.5$ & $h=0.25$\\[2em]
\multirow{2}{*}{\rotatebox{90}{$n$-neighborhood}} & 
\figTable{0.25}{figures/chapter5/flow/triangle/radius_3/mdca/elastica/len_pen_0.01000/jonctions_1/curve_segs_4/best/gs_1.00000/summary.pdf} &
\figTable{0.25}{figures/chapter5/flow/triangle/radius_3/mdca/elastica/len_pen_0.01000/jonctions_1/curve_segs_4/best/gs_0.50000/summary.pdf} &
\figTable{0.25}{figures/chapter5/flow/triangle/radius_3/mdca/elastica/len_pen_0.01000/jonctions_1/curve_segs_4/best/gs_0.25000/summary.pdf}\\
& \figTable{0.25}{figures/chapter5/flow/flower/radius_3/mdca/elastica/len_pen_0.01000/jonctions_1/curve_segs_4/best/gs_1.00000/summary.pdf} &
\figTable{0.25}{figures/chapter5/flow/flower/radius_3/mdca/elastica/len_pen_0.01000/jonctions_1/curve_segs_4/best/gs_0.50000/summary.pdf} &
\figTable{0.25}{figures/chapter5/flow/flower/radius_3/mdca/elastica/len_pen_0.01000/jonctions_1/curve_segs_4/best/gs_0.25000/summary.pdf}\\[8em]

\hline\\[2em]

\multirow{2}{*}{\rotatebox{90}{Extended $n$-neighborhood}} & 
\figTable{0.25}{figures/chapter5/mdca-larger-neighborhood/triangle/0.01/1.0/summary.pdf} &
\figTable{0.25}{figures/chapter5/mdca-larger-neighborhood/triangle/0.01/0.5/summary.pdf} &
\figTable{0.25}{figures/chapter5/mdca-larger-neighborhood/triangle/0.01/0.25/summary.pdf}\\

& \figTable{0.25}{figures/chapter5/mdca-larger-neighborhood/flower/0.01/1.0/summary.pdf} &
\figTable{0.25}{figures/chapter5/mdca-larger-neighborhood/flower/0.01/0.5/summary.pdf} &
\figTable{0.25}{figures/chapter5/mdca-larger-neighborhood/flower/0.01/0.25/summary.pdf}


\end{tabular}
\caption{In the top row, the MDCA evolution for the neighborhood as presented in~\cref{alg:local-search}. In the bottom row, the flow using the extended neighborhood. The extended neighborhood additionaly includes the $n$-neighborhood of the dilation and the erosion of the initial shape by a square of side $1$.}
\label{fig:mdca-larger-neighborhood}
\end{figure}



\begin{figure}[]
\begin{minipage}[b]{0.6\textwidth}
\center
\includegraphics[scale=0.15]{figures/chapter5/mdca-sensitivity/closer-picture.pdf}
\end{minipage}%
\begin{minipage}[b]{0.4\textwidth}
\center
\includegraphics[scale=0.025]{figures/chapter5/mdca-sensitivity/big-picture.pdf}\\\vspace{2em}
\captionsetup{type=table}
\begin{tabular}{r|c|c}
& II-$5$ & MDCA \\
\hline
Red  & 5.54 & 3.93\\
Blue & 5.55 & 3.84\\
\hline
$| \Delta E / \Delta \Ds |$ & 70 & 1400
\end{tabular}
\end{minipage}
\caption{A slight variation in the shape boundary (in this example, a $0.07\%$ change or $4$ pixels over $5310$) inflicts a considerably higher change in the energy value when using MDCA than when using II. }
\label{fig:mdca-sensitivity}
\end{figure}

\subsection{Constrained digital Elastica}
\label{ch6:subsec:constrained-digital-elastica}

An important advantage of~\cref{alg:local-search} is that constraints can be imposed with minimum effort. We present results for two types of constraints. In the first type, we force some pixels to be part of the final solution and in the second we impose orientations at the endpoints of a curve. In~\cref{fig:constrained-elastica} we compare the flows for different values of $\alpha$.

We remark that~\cref{alg:local-search} is sensitive to the parameter $\alpha$. For higher values of $\alpha$, the shapes tends to shrink and the curves are closer to a straight line. For lower values of $\alpha$, the shapes tends to grow and the curves make more turns. 

\begin{figure}
\center
\begin{tabular}{ccc}
$\alpha=0.1$ & $\alpha=0.01$ & $\alpha=0.001$\\[2em]
\includegraphics[scale=0.25]{figures/chapter5/fixed-pixels/elastica/len_pen_0.1/flower-1/summary.pdf} &
\includegraphics[scale=0.25]{figures/chapter5/fixed-pixels/elastica/len_pen_0.01/flower-1/summary.pdf} &
\includegraphics[scale=0.25]{figures/chapter5/fixed-pixels/elastica/len_pen_0.001/flower-1/summary.pdf}\\[2em]
\includegraphics[scale=0.25]{figures/chapter5/fixed-pixels/elastica/len_pen_0.1/flower-2/summary.pdf} &
\includegraphics[scale=0.25]{figures/chapter5/fixed-pixels/elastica/len_pen_0.01/flower-2/summary.pdf} &
\includegraphics[scale=0.25]{figures/chapter5/fixed-pixels/elastica/len_pen_0.001/flower-2/summary.pdf}\\[2em]
\includegraphics[scale=0.25]{figures/chapter5/fixed-orientations/elastica/len_pen_0.1/curve-2/summary.pdf} &
\includegraphics[scale=0.25]{figures/chapter5/fixed-orientations/elastica/len_pen_0.01/curve-2/summary.pdf} &
\includegraphics[scale=0.25]{figures/chapter5/fixed-orientations/elastica/len_pen_0.001/curve-2/summary.pdf}\\[2em]
\includegraphics[scale=0.25]{figures/chapter5/fixed-orientations/elastica/len_pen_0.1/curve-3/summary.pdf} &
\includegraphics[scale=0.25]{figures/chapter5/fixed-orientations/elastica/len_pen_0.01/curve-3/summary.pdf} &
\includegraphics[scale=0.25]{figures/chapter5/fixed-orientations/elastica/len_pen_0.001/curve-3/summary.pdf}
\end{tabular}
\caption{In the first and second rows, the flow obtained by forcing the green pixels to be part of the final solution; In the last two rows, the flow obtained by forcing the orientation at the endpoints of the curves.}
\label{fig:constrained-elastica}

\end{figure}

\subsection{Running time}
\label{ch6:subsec:running-time}

The running time of~\cref{alg:local-search} is summarized in table~\cref{tab:summary-local-comb-rtime}. All the experiments in this thesis were executed on a $32$-core $2.4Ghz$ CPU. Although its use in practical applications is
limited, we demonstrated that digital estimators are effective in their measurements and the flows evolve as expected, reaching the global optimum for some shapes. We
observe that it is a complete digital approach, and we do not suffer from discretization and rounding problems, a common
issue in continuous models.  Furthermore we have checked that this approach works indifferently with Integral Invariant
curvature estimator and Maximal Digital Circular Arc curvature estimator, given an appropriate neighborhood. So the convergence of the digital curvature
estimator seems to be the cornerstone to get a digital curve behaving like a continuous Elastica. 

\begin{figure}[h!]
\center
\captionsetup{type=table}
\begin{tabular}{|l|c|c|c|c|c|c|}
\hline
& \multicolumn{2}{c|}{$h=1.0$} & \multicolumn{2}{c|}{$h=0.5$} & \multicolumn{2}{c|}{$h=0.25$}\\
\hline
& Pixels & Time & Pixels & Time & Pixels & Time\\
\hline
Triangle & 521 & 2s (0.07s/it)  & 2080 & 43s (0.81s/it) & 8315 & 532s(4.8s/it)\\
Square & 841 & 0.9s (0.09s/it) & 3249 & 8s (0.3s/it) & 12769 & 102s (2s/it)\\
Flower & 1641 & 13s (0.24s/it) & 6577 & 209s (1.68s/it) & 26321 & 3534s (12.3s/it)\\
Bean  & 1574 & 7s (0.16s/it) & 6278 & 88s (1.08s/it) & 25130 & 1131s (6.4s/it)\\
Ellipse  & 626 & 1s (0.14s/it) & 2506 & 16s (0.44s/it) & 10038 & 286s (3.1s/it)\\
\hline
\end{tabular}
\caption{Running time of LocalSearch for the free Elastica problem.}
\label{tab:summary-local-comb-rtime} 
\end{figure}





\section{Global optimization}
\label{ch6:sec:global-optimization}

In this section we turn to a global optimization approach. However, instead of minimizing~\cref{eq:digital-elastica} we are going to optimize a simplified version of it in which we don't need to compute the local length estimator.

\subsection{Simplified digital Elastica}
\label{ch6:subsec:simplified-digital-elastica}

The simplified digital Elastica is defined as

	\begin{align}
	\hat{E}_{\Theta}^{simp}( D_h(S) ) = \sum_{\dot{\vec{e}} \in \partial D_h(S)}{ \alpha + \beta \hat{\kappa}_{r}^2(D_h(S),\dot{\vec{e}},h) }.
	\label{eq:simplified-digital-elastica}
	\end{align}
	

We argue that~\cref{eq:simplified-digital-elastica} is a reasonable approximation of~\cref{eq:digital-elastica}. Indeed, executing~\cref{alg:local-search} for minimize the simplified digital Elastica obtains very similar results to those for the digital Elastica (see~\cref{fig:simplified-elastica}).


\begin{figure}[]
\center
\begin{tabular}{ccc}
\includegraphics[scale=0.25]{figures/chapter5/flow/triangle/radius_5/ii/selastica/len_pen_0.01000/jonctions_1/curve_segs_4/best/gs_0.25000/summary.pdf} &
\includegraphics[scale=0.25]{figures/chapter5/flow/square/radius_5/ii/selastica/len_pen_0.01000/jonctions_1/curve_segs_4/best/gs_0.25000/summary.pdf} &
\includegraphics[scale=0.25]{figures/chapter5/flow/ellipse/radius_5/ii/selastica/len_pen_0.01000/jonctions_1/curve_segs_4/best/gs_0.25000/summary.pdf}\\[2em]
\includegraphics[scale=0.25]{figures/chapter5/flow/flower/radius_5/ii/selastica/len_pen_0.01000/jonctions_1/curve_segs_4/best/gs_0.25000/summary.pdf} &
\includegraphics[scale=0.25]{figures/chapter5/flow/bean/radius_5/ii/selastica/len_pen_0.01000/jonctions_1/curve_segs_4/best/gs_0.25000/summary.pdf} &
\includegraphics[scale=0.25]{figures/chapter5/fixed-pixels/selastica/len_pen_0.01/flower-1/summary.pdf}\\[2em]
\includegraphics[scale=0.25]{figures/chapter5/fixed-pixels/selastica/len_pen_0.01/flower-2/summary.pdf} &
\includegraphics[scale=0.25]{figures/chapter5/fixed-orientations/selastica/len_pen_0.01/curve-2/summary.pdf} &
\includegraphics[scale=0.25]{figures/chapter5/fixed-orientations/selastica/len_pen_0.01/curve-3/summary.pdf}
\end{tabular}
\caption{Experiments of~\cref{ch6:sec:local-combinatorial-scheme} for the simplified digital Elastica.}
\label{fig:simplified-elastica}
\end{figure}


\subsection{Optimization model for simplified digital Elastica}
\label{ch6:subsec:optimization-model-simplified-digital-elastica}

Differently from the previous section, the model described here is designed for the integral invariant estimator only. Let $\Ds \in \frac{1}{2}\mathbb{Z}^2$ be the digitization of some shape $S \in \mathbb{R}^2$ using grid step $h$ in half-integer coordinates space. We assume that $\Ds$ has $m$ pixels (located at integer coordinates) and $n$ linels (one and only one of its coordinates is $\frac{1}{2}$). Optimization variables are represented as column vectors $\vec{x} \in \mathbb{B}^{m},\, \vec{y} \in \mathbb{B}^{n}$ and its $i$-th coefficients are denoted  $\vec{x}_i,\vec{y}_i$.  Further, let $\vec{A} \in \mathbb{B}^{m\times n}$ the matrix defined as

\[
	\vec{A}_{i,j} = \left\{ \begin{array}{ll}
		1,\; x_j \in B_{r}(y_i)\\
		0,\; \text{otherwise}.
	\end{array}\right.
\]

In other words, the column vector $\vec{A}_i$ of $\vec{A}$ represents the pixels that are in the interior of the disk $B_{r}(y_i)$ of radius $r$ centered at $\vec{y}_i$. 


\begin{align}
	E_{\Theta}^{simp}(\vec{x},\vec{y}) =& \sum_{\vec{y}_i \in \vec{y}}{ \vec{y}_i \left(\; \alpha + \beta \hat{\kappa}_{r}^2(D,\vec{y}_i) \; \right)}\\\nonumber
			   =& \sum_{\vec{y}_i \in \vec{y}}{ \vec{y}_i \left(\; \alpha  + \beta \big( \frac{3}{r^3}(\frac{\pi}{r^2} - |B_r(\vec{y}_i)|)\big)^2\right)}\\\nonumber
			   =& \sum_{\vec{y}_i \in \vec{y}}{ \vec{y}_i \left(\; \alpha + \frac{9}{r^6}\beta \big(c^2 - 2c\vec{A}_i^T\vec{x} + \vec{x}^T\vec{A}_i\vec{A}_i^T\vec{x}\big)\right)},			   
	\end{align}
	
where $c =  \pi r^2/2$. We remark that linels and pixels in the solution must be topologicaly consistent, .i.e., linels must form connected closed curves and the pixels must lie in the interior of those curves. This restriction is encoded in a set of topological constraints $T(x,y)$ detailed later. So far we have

\begin{align*}
	\min_{\vec{x} \in \mathbb{B}^{|X|}, \vec{y} \in \mathbb{B}^{|Y|}}{E_{\Theta}^{simp}(\vec{x},\vec{y})}, \quad \text{subject to } T(\vec{x},\vec{y}). \quad (P0)
\end{align*}

Additionaly, in real applications involving the minimization of Elastica, we have a set of constraints $R$ that plays the role of regularization. For example, we may force some of the pixels in the original shape to be part of the solution; for imaging problems, we may add a data attachment term, and so on. Finally, we can write the general optimization problem as

\begin{align*}
	\min_{\vec{x} \in \mathbb{B}^{|X|}, \vec{y} \in \mathbb{B}^{|Y|}}{E_{\Theta}^{simp}(\vec{x},\vec{y})}, \quad \text{subject to } T(\vec{x},\vec{y}), R(x) \quad (P1)
\end{align*}

	Formulation $P1$ is a constrained binary non-convex third order problem and likely difficult to be solved optimally. Nonetheless, we can use standard optimization techniques to acquire some intuition on the model. 	
	
\subsection{Topological constraints}
\label{ch6:subsec:topological-constraints}

The estimation ball should be applied in the digital boundary of the shape, which oblige us to impose topological constraints in the model to avoid inconsistent solutions. In order to accomplish that, we set an arbitrary orientation for the faces and another for the edges. We choose counter-clockwise for faces; left-to-right for horizontal edges; and bottom-to-up for vertical edges.


We create the vector $\vec{z} \in \mathbb{B}^{2n}$. We map each linel identified by variable $\vec{y}_i$ to components $\vec{z}_{2i},\vec{z}_{2i+1}$, one for each possible orientation the linel may assume. Next, we extend the linel incidence matrix defined in~\cref{app:pixel-incidence-matrix} to hold incidence with respect to oriented edges. The new matrix $\vec{T} \in \mathbb{B}^{n \times m + 2n}$ is defined as

\[
	0 \leq j < m, \quad \vec{T}_{i,j} = \left\{ \begin{array}{ll}
	
	1,& \text{Pixel $j$ is positively incident to linel $i$}\\
	-1,& \text{Pixel $j$ is negatively incident to linel $i$}\\	
	0,& \text{otherwise},
	\end{array}\right.
\]

\[
	m \leq j < m + 2n, \quad \vec{T}_{i,j} = \left\{ \begin{array}{ll}
	
	1,& \text{Edge $j$ is positively incident to linel $i$}\\
	-1,& \text{Edge $j$ is negatively incident to linel $i$}\\	
	0,& \text{otherwise}.
	\end{array}\right.
\]

Rewriting formulation (P1)

\[
\begin{array}{ll}
& \displaystyle	\min \sum_{z_i \in \vec{z}}{ \vec{z}_i \left(\; \alpha + \frac{9}{r^6}\beta \big(c^2 - 2c\vec{A}_i^T\vec{x} + \vec{x}^T\vec{A}_i\vec{A}_i^T\vec{x}\big)\right)} \\
\text{subject to}\\
&	\vec{T} \times  \left[ \begin{array}{c}
							\vec{x} \\ 
							\vec{z} 
						   \end{array} \right] = 0 \\
&   R(\vec{x}),\\
&   \vec{x} \in \mathbb{B}^{m}, \vec{z} \in \mathbb{B}^{2n}.
\end{array}
\]


We observe that for a linel identified by variable $\vec{y}_i$, constraints $\vec{T}$ forces at most one of the variables $\vec{z}_{2i},\vec{z}_{2i+1}$ to be evaluated to one. 


\subsection{Linear relaxation of $P1$}
\label{ch6:subsec:linear-relaxation}

	The simplest model we can derive from (P1) consists in to relax the optimization variables, i.e., we impose $\vec{x} \in \mathbb{U}^m$ and $\vec{z} \in \mathbb{U}^{2n}$, and we linearize all second and third order terms. 
	
	Consider the summation in (P1). An opt-term is an ordered sequence of optimization variables, e.g., the opt-term $\vec{x}_2^2\vec{x}_4$ is encoded as the sequence $(\vec{x}_2,\vec{x}_2,\vec{x}_4)$. Let $\mathcal{T}$ the collection of opt-terms of order two or higher in (P1) and $\mathcal{T}_i \in \mathcal{T}$ a member of this collection. To linearize (P1), we associate a variable $\vec{u_i}$ for each term of $\mathcal{T}$, i.e., $\vec{u} \in \mathbb{U}^{|\mathcal{T}|}$ and we enforce $|\mathcal{T}_i|+1$ new constraints. In other words, we add the following set of linearization constraints.
	
\[
	L(\vec{u}) = \left\{ \Big\{ \vec{u}_i \leq t, \quad \forall t \in \mathcal{T}_i \Big\} \cup \Big\{ \vec{u}_i \geq \displaystyle \sum_{t \in \mathcal{T}_i}{t} - |\mathcal{T}_i| + 1 \Big\} \; \Big| \; \forall \mathcal{T}_i \in \mathcal{T} \right\}
\]

The linearization of (P1) is written as

\[
\begin{array}{ll}
& \displaystyle	\min \sum_{z_i \in \vec{z}}{ \vec{z}_i \left(\; \alpha + \frac{9}{r^6}\beta \big(c^2 - 2c\vec{A}_i^T\vec{x} + \vec{x}^T\vec{A}_i\vec{A}_i^T\vec{x}\big)\right)} \\
\text{subject to}\\
&	\vec{T} \times  \left[ \begin{array}{c}
							\vec{x} \\ 
							\vec{z} 
						   \end{array} \right] = 0 \\
&   R(\vec{x}),\\
&   L(\vec{u}),\\
&   \vec{x} \in \mathbb{U}^{m}, \vec{z} \in \mathbb{U}^{2n}, \vec{u} \in \mathbb{U}^{|\mathcal{T}|} 
\end{array}
\]

	Finally, to obtain a binary vector we round the partial solution vector $\vec{x^{\star}} \in \mathbb{U}^m$. For an instance with $m$ pixels we have about $2m$ linels. After linearization, we can expect to have up to $O(m^3)$ variables, dampening our attempts to solve it globally even for low resolution images. One can also try quadratic formulations by linearizing only the third order terms. Unfortunately, the matrix of quadratic terms is not semi-definite positive, fundamental condition for efficient optimization of the model.
	


\subsection{Unconstrained version of P1}
\label{ch6:subsec:unconstrained-version}

We can use the pixel incidence matrix defined in~\cref{app:pixel-incidence-matrix} to define an unconstrained version of P1. The pixel incidence vector $\vec{q} \in \mathbb{Z}^m$ for pixels $\vec{x} \in \mathbb{B}^{m}$ is 
	
	\begin{align*}
		\vec{q} &= \vec{P}^T\vec{P} \vec{x}
	\end{align*}

In order to supress the sign, we define diagonal matrix $\vec{Q} \in \mathbb{R}^{m \times m }$ as

\begin{align*}
	\vec{Q} = diag(\vec{q})diag(\vec{q})
\end{align*}

Let $\vec{B} \in \mathbb{B}^{m\times m}$ such that column vector $\vec{B}_j$ represents the pixels in the interior of a disk of radius $R$ centered at pixel $i$. Finally, we search for solutions of

\begin{align}
	\min_{\vec{x}} \frac{9}{R^6}\sum_{j}^{m}{\left( \frac{\pi R^2}{2} - \frac{1}{2}\boldsymbol{\mathbf{1}}^T{\vec{Q}}\vec{B}_j \right)^2},
	\label{eq:unconstrained-digital-elastica}
\end{align}


where $\mathbf{1} = (1,1, \cdots , 1)^T \in \mathbb{R}^m$. ~\cref{eq:unconstrained-digital-elastica} involves the minimization of a fourth order equation and therefore hard to be optimized.

\section{Conclusion}
\label{ch6:sec:conclusion}
We gave a historical review of the Elastica and we defined the digital Elastica energy in~\cref{ch6:sec:continuous-digital-elastica}. The local combinatorial scheme defined in~\cref{ch6:sec:local-combinatorial-scheme} can evolve different shapes guided by the minimization of digital Elastica energy and it eventually reaches global optimum in the free Elastica problem, justifying the interest for multigrid convergent estimators. The model can also be used to solve the constrained Elastica problem, but its more likely to stop in a local optimum. Finnaly, we sketch some global optimization models in~\cref{ch6:sec:global-optimization} for minimizing the simplified Elastica using standard techniques of optimization. The difficulties we pointed out suggest that a practical global optimization model is unlikely to exist. In the next chapter we explore a model that decreases the Elastica energy and that can be used in practice.

