\chapter{Conclusion and perspectives}
\label{chapter:conclusion-and-perspectives}

The goal of this thesis was to study and propose models for image processing problems using multigrid convergent estimators of geometric properties, in particular the curvature and the minimization of the elastica energy. We argued that classical discretization schemes are based in the assumption that an exact sampling is available, which is not the case for digital images. The idea that convergence to the estimated value is achieved as the number of considered points increases is no longer valid in the digital world. A trivial extension of the linear discretization of~\cref{ch3:sec:discrete-methods-squared-curvature} to the digital space would consider all the digital pixels, but in this case, only angle variations of $90$ degrees would be measured.

Our choice for the minimization of elastica was motivated by the completion property and its applications in both inpainting and segmentation. It is a challenging problem and the current models are limited to the completion of small regions and/or have strong metrication errors. One reason for that is that the completion property is quite difficult to be recovered by local approaches. Nonetheless, we have insisted on those since our attempts of global optimization happened to be impractical.

Another difficult is to inject the multigrid convergent estimators in an optimization framework. Estimators as MDCA and MDSS are based on the computation of digital geometry primitives whose expression as an energy to be optimized is not trivial to be derived. The most suitable to this task is indeed the II estimator, for which we tested global optimization models, a constrained and an unconstrained one, but without success.

The first model we studied, the LocalSearch model, had the goal to validate the use of  multigrid convergent estimators in an optimization framework, which in this sense, we were successful. In the LS model, there is no restriction with respect to the estimators used. We achieved the global optimum in several occasions for the free elastica problem, and in the constrained elastica, the final curves resembled those minimizing the elastica. This model finds fewer applications due its high running time, but it was used as a ground truth for the others that followed.

The BalanceFlow was developed for the II estimator and it is based in the concept of balance coefficient. It builds up on the FlipFlow model and it can be considered as an improvement of the latter, though the models are not exactly the same. The running times are much faster and we have proposed an application to image segmentation. However, they present a shrinking bias that make them resemble more the curve-shortening flow than the elastica flow. Consequently, the elastica is decreased until a certain inflection point.

Finally, the GraphFlow model exploits strategies present in all the previous models to produce our fastest algorithm. The balance coefficient is used to ponder the edges of the candidate graphs which are derived from a neighborhood of shapes: the $a$-probe set. The flows produced in the free elastica problem behaves as expected and, like the LocalSearch model, it achieves the global optimum in several occasions. We demonstrated its use in image segmentation and we were able to recover the completion property, but only for small regions.

\subsubsection{Perspectives}

\begin{itemize}
	\item[]{\textbf{GraphFlow and perimeter.} In the candidate selection step, the candidate graphs are pondered with their balance coefficients and data term(s) from image input in the case of segmentation, but we have not tested to also add a perimeter cost, for example, the costs proposed in~\cite{boykov03geodesics}}. 
	\item[]{\textbf{GraphFlow and random neighborhood of shapes.} The $a$-probe set is a very simple neighborhood and not appropriate in some scenarios, as demonstrated in the contrained elastica problem. We could take advantage of the fact that the candidate graphs are relatively small and that the candidate selection step could be implemented in parallel to devise a stronger algorithm without considerable gain in running time.} 
	\item[]{\textbf{Dynamic radius.} We have seen that the radius of the estimation disk should not be set to a value higher than the reach of the shape, but we also observe that a too small radius is not sensitive enough to subtle variations in a region of low curvature. A dynamic approach could be implemented with the help of the MDCA estimator, for example, which is a parameter-free estimator. }
	\item[]{\textbf{Multiresolution.} We can improve the running time of segmentation in large images by employing a multiresolution approach. Moreover, the multigrid convergence property give us more guarantees of estimation quality in different resolutions and it could be very useful in strategies of this type.}
	\item[]{\textbf{Global optimization and multigrid convergent estimators.} A global approach is crucial to recover the completion property in its totality, and a discrete global model using multigrid convergent estimator of curvature is still unknown.}
\end{itemize}
 




%\subsubsection{Notes}
%
%\begin{itemize}
%\item[]{\textbf{Capitulo 5.} A versao simplificada do digital elastica pode servir de argumento para justificar que usar estimadores multigrid convergentes nao eh tao necessaria assim.}
%\item[]{\textbf{Capitulo 6.} 1) Por que nao usar um estimador multigrid convergente para o perimetro no FlipFlow? Comentar as dificuldades de se implementar os estimadores em um framework de otimizacao, e.g. MDCA e MDSS; 2) Deveria mencionar a capacidade do metodo no que diz respeito a propriedade de completude.  }
%\item[]{\textbf{Capitulo 8.} 1) Existem outras alternativas para a funcao de validacao do GraphFlow que nao aquela derivada do modelo original de graph-cuts; 2) Eu menciono shape reach mas nunca defino o que seja; 2) A fase de selecao de candidatos nao coloca penalizacao no perimetro. Poderia-se criar uma versao em que se usa o peso do geodesics via graph cut.}
%\item[]{\textbf{Geral.} Avaliar se existe necessidade de se fazer uma distincao entra imagens coloridades e em escala de cinza.}
%\end{itemize}